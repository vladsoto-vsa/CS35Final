\par\begingroup
\newcommand{\treeDiagramWidth}{0.9\textwidth}
\newcommand{\minimumNodeSize}{8mm}
\newcommand{\verticalSpacing}{10mm}
\begin{center}\begin{tikzpicture}
\tikzstyle{treeNode}=[minimum width=\minimumNodeSize,minimum height=\minimumNodeSize,circle,draw,inner sep=1mm]
\tikzstyle{treeEdge}=[->]
\node[minimum width=\treeDiagramWidth,rectangle] at (\treeDiagramWidth/2,0) {};\node[treeNode] (n-0-0) at ( {(\treeDiagramWidth-(\minimumNodeSize*1))/(1)*(0+0.5) + \minimumNodeSize/2 + \minimumNodeSize*0}, {-0*\verticalSpacing}) {30};
\node[treeNode] (n-1-0) at ( {(\treeDiagramWidth-(\minimumNodeSize*2))/(2)*(0+0.5) + \minimumNodeSize/2 + \minimumNodeSize*0}, {-1*\verticalSpacing}) {20};
\node[treeNode] (n-2-0) at ( {(\treeDiagramWidth-(\minimumNodeSize*4))/(4)*(0+0.5) + \minimumNodeSize/2 + \minimumNodeSize*0}, {-2*\verticalSpacing}) {10};
\draw[treeEdge] (n-1-0) -- (n-2-0);
\draw[treeEdge] (n-0-0) -- (n-1-0);
\node[treeNode] (n-1-1) at ( {(\treeDiagramWidth-(\minimumNodeSize*2))/(2)*(1+0.5) + \minimumNodeSize/2 + \minimumNodeSize*1}, {-1*\verticalSpacing}) {40};
\draw[treeEdge] (n-0-0) -- (n-1-1);
\end{tikzpicture}\end{center}
\endgroup\par

Since 12 is smaller than the root, 30, it goes to the left
12 becomes leaf node


\par\begingroup
\newcommand{\treeDiagramWidth}{0.9\textwidth}
\newcommand{\minimumNodeSize}{8mm}
\newcommand{\verticalSpacing}{10mm}
\begin{center}\begin{tikzpicture}
\tikzstyle{treeNode}=[minimum width=\minimumNodeSize,minimum height=\minimumNodeSize,circle,draw,inner sep=1mm]
\tikzstyle{treeEdge}=[->]
\node[minimum width=\treeDiagramWidth,rectangle] at (\treeDiagramWidth/2,0) {};\node[treeNode] (n-0-0) at ( {(\treeDiagramWidth-(\minimumNodeSize*1))/(1)*(0+0.5) + \minimumNodeSize/2 + \minimumNodeSize*0}, {-0*\verticalSpacing}) {30};
\node[treeNode] (n-1-0) at ( {(\treeDiagramWidth-(\minimumNodeSize*2))/(2)*(0+0.5) + \minimumNodeSize/2 + \minimumNodeSize*0}, {-1*\verticalSpacing}) {20};
\node[treeNode] (n-2-0) at ( {(\treeDiagramWidth-(\minimumNodeSize*4))/(4)*(0+0.5) + \minimumNodeSize/2 + \minimumNodeSize*0}, {-2*\verticalSpacing}) {10};
\node[treeNode] (n-3-1) at ( {(\treeDiagramWidth-(\minimumNodeSize*8))/(8)*(1+0.5) + \minimumNodeSize/2 + \minimumNodeSize*1}, {-3*\verticalSpacing}) {12};
\node[treeNode] (n-4-3) at ( {(\treeDiagramWidth-(\minimumNodeSize*16))/(16)*(3+0.5) + \minimumNodeSize/2 + \minimumNodeSize*3}, {-4*\verticalSpacing}) {30};
\node[treeNode] (n-5-6) at ( {(\treeDiagramWidth-(\minimumNodeSize*32))/(32)*(6+0.5) + \minimumNodeSize/2 + \minimumNodeSize*6}, {-5*\verticalSpacing}) {10};
\node[treeNode] (n-6-13) at ( {(\treeDiagramWidth-(\minimumNodeSize*64))/(64)*(13+0.5) + \minimumNodeSize/2 + \minimumNodeSize*13}, {-6*\verticalSpacing}) {20};
\node[treeNode] (n-7-26) at ( {(\treeDiagramWidth-(\minimumNodeSize*128))/(128)*(26+0.5) + \minimumNodeSize/2 + \minimumNodeSize*26}, {-7*\verticalSpacing}) {12};
\draw[treeEdge] (n-6-13) -- (n-7-26);
\draw[treeEdge] (n-5-6) -- (n-6-13);
\draw[treeEdge] (n-4-3) -- (n-5-6);
\node[treeNode] (n-5-7) at ( {(\treeDiagramWidth-(\minimumNodeSize*32))/(32)*(7+0.5) + \minimumNodeSize/2 + \minimumNodeSize*7}, {-5*\verticalSpacing}) {40};
\draw[treeEdge] (n-4-3) -- (n-5-7);
\draw[treeEdge] (n-3-1) -- (n-4-3);
\draw[treeEdge] (n-2-0) -- (n-3-1);
\draw[treeEdge] (n-1-0) -- (n-2-0);
\draw[treeEdge] (n-0-0) -- (n-1-0);
\node[treeNode] (n-1-1) at ( {(\treeDiagramWidth-(\minimumNodeSize*2))/(2)*(1+0.5) + \minimumNodeSize/2 + \minimumNodeSize*1}, {-1*\verticalSpacing}) {40};
\draw[treeEdge] (n-0-0) -- (n-1-1);
\end{tikzpicture}\end{center}
\endgroup\par

Since it's inbalanced and LSTH - RSTH > 1
right rotation is performed


\par\begingroup
\newcommand{\treeDiagramWidth}{0.9\textwidth}
\newcommand{\minimumNodeSize}{8mm}
\newcommand{\verticalSpacing}{10mm}
\begin{center}\begin{tikzpicture}
\tikzstyle{treeNode}=[minimum width=\minimumNodeSize,minimum height=\minimumNodeSize,circle,draw,inner sep=1mm]
\tikzstyle{treeEdge}=[->]
\node[minimum width=\treeDiagramWidth,rectangle] at (\treeDiagramWidth/2,0) {};\node[treeNode] (n-0-0) at ( {(\treeDiagramWidth-(\minimumNodeSize*1))/(1)*(0+0.5) + \minimumNodeSize/2 + \minimumNodeSize*0}, {-0*\verticalSpacing}) {20};
\node[treeNode] (n-1-0) at ( {(\treeDiagramWidth-(\minimumNodeSize*2))/(2)*(0+0.5) + \minimumNodeSize/2 + \minimumNodeSize*0}, {-1*\verticalSpacing}) {10};
\node[treeNode] (n-2-1) at ( {(\treeDiagramWidth-(\minimumNodeSize*4))/(4)*(1+0.5) + \minimumNodeSize/2 + \minimumNodeSize*1}, {-2*\verticalSpacing}) {12};
\node[treeNode] (n-3-3) at ( {(\treeDiagramWidth-(\minimumNodeSize*8))/(8)*(3+0.5) + \minimumNodeSize/2 + \minimumNodeSize*3}, {-3*\verticalSpacing}) {30};
\node[treeNode] (n-4-6) at ( {(\treeDiagramWidth-(\minimumNodeSize*16))/(16)*(6+0.5) + \minimumNodeSize/2 + \minimumNodeSize*6}, {-4*\verticalSpacing}) {12};
\node[treeNode] (n-5-12) at ( {(\treeDiagramWidth-(\minimumNodeSize*32))/(32)*(12+0.5) + \minimumNodeSize/2 + \minimumNodeSize*12}, {-5*\verticalSpacing}) {10};
\draw[treeEdge] (n-4-6) -- (n-5-12);
\node[treeNode] (n-5-13) at ( {(\treeDiagramWidth-(\minimumNodeSize*32))/(32)*(13+0.5) + \minimumNodeSize/2 + \minimumNodeSize*13}, {-5*\verticalSpacing}) {20};
\draw[treeEdge] (n-4-6) -- (n-5-13);
\draw[treeEdge] (n-3-3) -- (n-4-6);
\node[treeNode] (n-4-7) at ( {(\treeDiagramWidth-(\minimumNodeSize*16))/(16)*(7+0.5) + \minimumNodeSize/2 + \minimumNodeSize*7}, {-4*\verticalSpacing}) {40};
\draw[treeEdge] (n-3-3) -- (n-4-7);
\draw[treeEdge] (n-2-1) -- (n-3-3);
\draw[treeEdge] (n-1-0) -- (n-2-1);
\draw[treeEdge] (n-0-0) -- (n-1-0);
\node[treeNode] (n-1-1) at ( {(\treeDiagramWidth-(\minimumNodeSize*2))/(2)*(1+0.5) + \minimumNodeSize/2 + \minimumNodeSize*1}, {-1*\verticalSpacing}) {30};
\node[treeNode] (n-2-3) at ( {(\treeDiagramWidth-(\minimumNodeSize*4))/(4)*(3+0.5) + \minimumNodeSize/2 + \minimumNodeSize*3}, {-2*\verticalSpacing}) {40};
\draw[treeEdge] (n-1-1) -- (n-2-3);
\draw[treeEdge] (n-0-0) -- (n-1-1);
\end{tikzpicture}\end{center}
\endgroup\par
