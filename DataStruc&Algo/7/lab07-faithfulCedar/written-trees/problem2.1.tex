\par\begingroup
\newcommand{\treeDiagramWidth}{0.9\textwidth}
\newcommand{\minimumNodeSize}{8mm}
\newcommand{\verticalSpacing}{10mm}
\begin{center}\begin{tikzpicture}
\tikzstyle{treeNode}=[minimum width=\minimumNodeSize,minimum height=\minimumNodeSize,circle,draw,inner sep=1mm]
\tikzstyle{treeEdge}=[->]
\node[minimum width=\treeDiagramWidth,rectangle] at (\treeDiagramWidth/2,0) {};\node[treeNode] (n-0-0) at ( {(\treeDiagramWidth-(\minimumNodeSize*1))/(1)*(0+0.5) + \minimumNodeSize/2 + \minimumNodeSize*0}, {-0*\verticalSpacing}) {9};
\node[treeNode] (n-1-0) at ( {(\treeDiagramWidth-(\minimumNodeSize*2))/(2)*(0+0.5) + \minimumNodeSize/2 + \minimumNodeSize*0}, {-1*\verticalSpacing}) {7};
\node[treeNode] (n-2-0) at ( {(\treeDiagramWidth-(\minimumNodeSize*4))/(4)*(0+0.5) + \minimumNodeSize/2 + \minimumNodeSize*0}, {-2*\verticalSpacing}) {4};
\node[treeNode] (n-3-0) at ( {(\treeDiagramWidth-(\minimumNodeSize*8))/(8)*(0+0.5) + \minimumNodeSize/2 + \minimumNodeSize*0}, {-3*\verticalSpacing}) {3};
\draw[treeEdge] (n-2-0) -- (n-3-0);
\node[treeNode] (n-3-1) at ( {(\treeDiagramWidth-(\minimumNodeSize*8))/(8)*(1+0.5) + \minimumNodeSize/2 + \minimumNodeSize*1}, {-3*\verticalSpacing}) {2};
\draw[treeEdge] (n-2-0) -- (n-3-1);
\draw[treeEdge] (n-1-0) -- (n-2-0);
\node[treeNode] (n-2-1) at ( {(\treeDiagramWidth-(\minimumNodeSize*4))/(4)*(1+0.5) + \minimumNodeSize/2 + \minimumNodeSize*1}, {-2*\verticalSpacing}) {5};
\node[treeNode] (n-3-2) at ( {(\treeDiagramWidth-(\minimumNodeSize*8))/(8)*(2+0.5) + \minimumNodeSize/2 + \minimumNodeSize*2}, {-3*\verticalSpacing}) {1};
\draw[treeEdge] (n-2-1) -- (n-3-2);
\draw[treeEdge] (n-1-0) -- (n-2-1);
\draw[treeEdge] (n-0-0) -- (n-1-0);
\node[treeNode] (n-1-1) at ( {(\treeDiagramWidth-(\minimumNodeSize*2))/(2)*(1+0.5) + \minimumNodeSize/2 + \minimumNodeSize*1}, {-1*\verticalSpacing}) {5};
\node[treeNode] (n-2-2) at ( {(\treeDiagramWidth-(\minimumNodeSize*4))/(4)*(2+0.5) + \minimumNodeSize/2 + \minimumNodeSize*2}, {-2*\verticalSpacing}) {3};
\draw[treeEdge] (n-1-1) -- (n-2-2);
\node[treeNode] (n-2-3) at ( {(\treeDiagramWidth-(\minimumNodeSize*4))/(4)*(3+0.5) + \minimumNodeSize/2 + \minimumNodeSize*3}, {-2*\verticalSpacing}) {2};
\draw[treeEdge] (n-1-1) -- (n-2-3);
\draw[treeEdge] (n-0-0) -- (n-1-1);
\end{tikzpicture}\end{center}
\endgroup\par

Since the BT has to remain complete, 9 will be inserted
as the right child of the left 5.


\par\begingroup
\newcommand{\treeDiagramWidth}{0.9\textwidth}
\newcommand{\minimumNodeSize}{8mm}
\newcommand{\verticalSpacing}{10mm}
\begin{center}\begin{tikzpicture}
\tikzstyle{treeNode}=[minimum width=\minimumNodeSize,minimum height=\minimumNodeSize,circle,draw,inner sep=1mm]
\tikzstyle{treeEdge}=[->]
\node[minimum width=\treeDiagramWidth,rectangle] at (\treeDiagramWidth/2,0) {};\node[treeNode] (n-0-0) at ( {(\treeDiagramWidth-(\minimumNodeSize*1))/(1)*(0+0.5) + \minimumNodeSize/2 + \minimumNodeSize*0}, {-0*\verticalSpacing}) {9};
\node[treeNode] (n-1-0) at ( {(\treeDiagramWidth-(\minimumNodeSize*2))/(2)*(0+0.5) + \minimumNodeSize/2 + \minimumNodeSize*0}, {-1*\verticalSpacing}) {7};
\node[treeNode] (n-2-0) at ( {(\treeDiagramWidth-(\minimumNodeSize*4))/(4)*(0+0.5) + \minimumNodeSize/2 + \minimumNodeSize*0}, {-2*\verticalSpacing}) {4};
\node[treeNode] (n-3-0) at ( {(\treeDiagramWidth-(\minimumNodeSize*8))/(8)*(0+0.5) + \minimumNodeSize/2 + \minimumNodeSize*0}, {-3*\verticalSpacing}) {3};
\draw[treeEdge] (n-2-0) -- (n-3-0);
\node[treeNode] (n-3-1) at ( {(\treeDiagramWidth-(\minimumNodeSize*8))/(8)*(1+0.5) + \minimumNodeSize/2 + \minimumNodeSize*1}, {-3*\verticalSpacing}) {2};
\draw[treeEdge] (n-2-0) -- (n-3-1);
\draw[treeEdge] (n-1-0) -- (n-2-0);
\node[treeNode] (n-2-1) at ( {(\treeDiagramWidth-(\minimumNodeSize*4))/(4)*(1+0.5) + \minimumNodeSize/2 + \minimumNodeSize*1}, {-2*\verticalSpacing}) {5};
\node[treeNode] (n-3-2) at ( {(\treeDiagramWidth-(\minimumNodeSize*8))/(8)*(2+0.5) + \minimumNodeSize/2 + \minimumNodeSize*2}, {-3*\verticalSpacing}) {1};
\draw[treeEdge] (n-2-1) -- (n-3-2);
\node[treeNode] (n-3-3) at ( {(\treeDiagramWidth-(\minimumNodeSize*8))/(8)*(3+0.5) + \minimumNodeSize/2 + \minimumNodeSize*3}, {-3*\verticalSpacing}) {9};
\draw[treeEdge] (n-2-1) -- (n-3-3);
\draw[treeEdge] (n-1-0) -- (n-2-1);
\draw[treeEdge] (n-0-0) -- (n-1-0);
\node[treeNode] (n-1-1) at ( {(\treeDiagramWidth-(\minimumNodeSize*2))/(2)*(1+0.5) + \minimumNodeSize/2 + \minimumNodeSize*1}, {-1*\verticalSpacing}) {5};
\node[treeNode] (n-2-2) at ( {(\treeDiagramWidth-(\minimumNodeSize*4))/(4)*(2+0.5) + \minimumNodeSize/2 + \minimumNodeSize*2}, {-2*\verticalSpacing}) {3};
\draw[treeEdge] (n-1-1) -- (n-2-2);
\node[treeNode] (n-2-3) at ( {(\treeDiagramWidth-(\minimumNodeSize*4))/(4)*(3+0.5) + \minimumNodeSize/2 + \minimumNodeSize*3}, {-2*\verticalSpacing}) {2};
\draw[treeEdge] (n-1-1) -- (n-2-3);
\draw[treeEdge] (n-0-0) -- (n-1-1);
\end{tikzpicture}\end{center}
\endgroup\par

The max-heap invariant must be met, so 9 is swapped with 5
since 9 > 5.


\par\begingroup
\newcommand{\treeDiagramWidth}{0.9\textwidth}
\newcommand{\minimumNodeSize}{8mm}
\newcommand{\verticalSpacing}{10mm}
\begin{center}\begin{tikzpicture}
\tikzstyle{treeNode}=[minimum width=\minimumNodeSize,minimum height=\minimumNodeSize,circle,draw,inner sep=1mm]
\tikzstyle{treeEdge}=[->]
\node[minimum width=\treeDiagramWidth,rectangle] at (\treeDiagramWidth/2,0) {};\node[treeNode] (n-0-0) at ( {(\treeDiagramWidth-(\minimumNodeSize*1))/(1)*(0+0.5) + \minimumNodeSize/2 + \minimumNodeSize*0}, {-0*\verticalSpacing}) {9};
\node[treeNode] (n-1-0) at ( {(\treeDiagramWidth-(\minimumNodeSize*2))/(2)*(0+0.5) + \minimumNodeSize/2 + \minimumNodeSize*0}, {-1*\verticalSpacing}) {7};
\node[treeNode] (n-2-0) at ( {(\treeDiagramWidth-(\minimumNodeSize*4))/(4)*(0+0.5) + \minimumNodeSize/2 + \minimumNodeSize*0}, {-2*\verticalSpacing}) {4};
\node[treeNode] (n-3-0) at ( {(\treeDiagramWidth-(\minimumNodeSize*8))/(8)*(0+0.5) + \minimumNodeSize/2 + \minimumNodeSize*0}, {-3*\verticalSpacing}) {3};
\draw[treeEdge] (n-2-0) -- (n-3-0);
\node[treeNode] (n-3-1) at ( {(\treeDiagramWidth-(\minimumNodeSize*8))/(8)*(1+0.5) + \minimumNodeSize/2 + \minimumNodeSize*1}, {-3*\verticalSpacing}) {2};
\draw[treeEdge] (n-2-0) -- (n-3-1);
\draw[treeEdge] (n-1-0) -- (n-2-0);
\node[treeNode] (n-2-1) at ( {(\treeDiagramWidth-(\minimumNodeSize*4))/(4)*(1+0.5) + \minimumNodeSize/2 + \minimumNodeSize*1}, {-2*\verticalSpacing}) {9};
\node[treeNode] (n-3-2) at ( {(\treeDiagramWidth-(\minimumNodeSize*8))/(8)*(2+0.5) + \minimumNodeSize/2 + \minimumNodeSize*2}, {-3*\verticalSpacing}) {1};
\draw[treeEdge] (n-2-1) -- (n-3-2);
\node[treeNode] (n-3-3) at ( {(\treeDiagramWidth-(\minimumNodeSize*8))/(8)*(3+0.5) + \minimumNodeSize/2 + \minimumNodeSize*3}, {-3*\verticalSpacing}) {5};
\draw[treeEdge] (n-2-1) -- (n-3-3);
\draw[treeEdge] (n-1-0) -- (n-2-1);
\draw[treeEdge] (n-0-0) -- (n-1-0);
\node[treeNode] (n-1-1) at ( {(\treeDiagramWidth-(\minimumNodeSize*2))/(2)*(1+0.5) + \minimumNodeSize/2 + \minimumNodeSize*1}, {-1*\verticalSpacing}) {5};
\node[treeNode] (n-2-2) at ( {(\treeDiagramWidth-(\minimumNodeSize*4))/(4)*(2+0.5) + \minimumNodeSize/2 + \minimumNodeSize*2}, {-2*\verticalSpacing}) {3};
\draw[treeEdge] (n-1-1) -- (n-2-2);
\node[treeNode] (n-2-3) at ( {(\treeDiagramWidth-(\minimumNodeSize*4))/(4)*(3+0.5) + \minimumNodeSize/2 + \minimumNodeSize*3}, {-2*\verticalSpacing}) {2};
\draw[treeEdge] (n-1-1) -- (n-2-3);
\draw[treeEdge] (n-0-0) -- (n-1-1);
\end{tikzpicture}\end{center}
\endgroup\par

Since 9 is now the child of 7, it must also swap.
This leaves 9 to be a child of the root 9, which meets the heap invariant.


\par\begingroup
\newcommand{\treeDiagramWidth}{0.9\textwidth}
\newcommand{\minimumNodeSize}{8mm}
\newcommand{\verticalSpacing}{10mm}
\begin{center}\begin{tikzpicture}
\tikzstyle{treeNode}=[minimum width=\minimumNodeSize,minimum height=\minimumNodeSize,circle,draw,inner sep=1mm]
\tikzstyle{treeEdge}=[->]
\node[minimum width=\treeDiagramWidth,rectangle] at (\treeDiagramWidth/2,0) {};\node[treeNode] (n-0-0) at ( {(\treeDiagramWidth-(\minimumNodeSize*1))/(1)*(0+0.5) + \minimumNodeSize/2 + \minimumNodeSize*0}, {-0*\verticalSpacing}) {9};
\node[treeNode] (n-1-0) at ( {(\treeDiagramWidth-(\minimumNodeSize*2))/(2)*(0+0.5) + \minimumNodeSize/2 + \minimumNodeSize*0}, {-1*\verticalSpacing}) {9};
\node[treeNode] (n-2-0) at ( {(\treeDiagramWidth-(\minimumNodeSize*4))/(4)*(0+0.5) + \minimumNodeSize/2 + \minimumNodeSize*0}, {-2*\verticalSpacing}) {4};
\node[treeNode] (n-3-0) at ( {(\treeDiagramWidth-(\minimumNodeSize*8))/(8)*(0+0.5) + \minimumNodeSize/2 + \minimumNodeSize*0}, {-3*\verticalSpacing}) {3};
\draw[treeEdge] (n-2-0) -- (n-3-0);
\node[treeNode] (n-3-1) at ( {(\treeDiagramWidth-(\minimumNodeSize*8))/(8)*(1+0.5) + \minimumNodeSize/2 + \minimumNodeSize*1}, {-3*\verticalSpacing}) {2};
\draw[treeEdge] (n-2-0) -- (n-3-1);
\draw[treeEdge] (n-1-0) -- (n-2-0);
\node[treeNode] (n-2-1) at ( {(\treeDiagramWidth-(\minimumNodeSize*4))/(4)*(1+0.5) + \minimumNodeSize/2 + \minimumNodeSize*1}, {-2*\verticalSpacing}) {7};
\node[treeNode] (n-3-2) at ( {(\treeDiagramWidth-(\minimumNodeSize*8))/(8)*(2+0.5) + \minimumNodeSize/2 + \minimumNodeSize*2}, {-3*\verticalSpacing}) {1};
\draw[treeEdge] (n-2-1) -- (n-3-2);
\node[treeNode] (n-3-3) at ( {(\treeDiagramWidth-(\minimumNodeSize*8))/(8)*(3+0.5) + \minimumNodeSize/2 + \minimumNodeSize*3}, {-3*\verticalSpacing}) {5};
\draw[treeEdge] (n-2-1) -- (n-3-3);
\draw[treeEdge] (n-1-0) -- (n-2-1);
\draw[treeEdge] (n-0-0) -- (n-1-0);
\node[treeNode] (n-1-1) at ( {(\treeDiagramWidth-(\minimumNodeSize*2))/(2)*(1+0.5) + \minimumNodeSize/2 + \minimumNodeSize*1}, {-1*\verticalSpacing}) {5};
\node[treeNode] (n-2-2) at ( {(\treeDiagramWidth-(\minimumNodeSize*4))/(4)*(2+0.5) + \minimumNodeSize/2 + \minimumNodeSize*2}, {-2*\verticalSpacing}) {3};
\draw[treeEdge] (n-1-1) -- (n-2-2);
\node[treeNode] (n-2-3) at ( {(\treeDiagramWidth-(\minimumNodeSize*4))/(4)*(3+0.5) + \minimumNodeSize/2 + \minimumNodeSize*3}, {-2*\verticalSpacing}) {2};
\draw[treeEdge] (n-1-1) -- (n-2-3);
\draw[treeEdge] (n-0-0) -- (n-1-1);
\end{tikzpicture}\end{center}
\endgroup\par
