% This LaTeX file contains your written lab questions.  You may answer these
% questions just by inserting your answer into this document.  You are not
% *required* to do your homework in LaTeX, but it's quite likely to be easier
% than e.g. the equation editor in OpenOffice Writer or Microsoft Word.
%
% If you're unfamiliar with LaTeX, see the document LearningLaTeX.tex in this
% same directory.  It contains a brief explanation and a few snippets of LaTeX
% code to get you started; in fact, it should have everything you need to
% complete this assignment.
\documentclass{article}

\usepackage{amsmath}
\usepackage{amssymb}
\usepackage{algpseudocode}
\usepackage{algorithmicx}
\usepackage{tikz}

\begin{document}

\section{AVL Trees}

\noindent \textbf{Problem 1.} Perform a left rotation on the root of the following tree.  Be sure to specify the X, Y, and Z subtrees used in the rotation.

\par\begingroup
\newcommand{\treeDiagramWidth}{0.9\textwidth}
\newcommand{\minimumNodeSize}{8mm}
\newcommand{\verticalSpacing}{10mm}
\begin{center}\begin{tikzpicture}
\tikzstyle{treeNode}=[minimum width=\minimumNodeSize,minimum height=\minimumNodeSize,circle,draw,inner sep=1mm]
\tikzstyle{treeEdge}=[->]
\node[minimum width=\treeDiagramWidth,rectangle] at (\treeDiagramWidth/2,0) {};\node[treeNode] (n-0-0) at ( {(\treeDiagramWidth-(\minimumNodeSize*1))/(1)*(0+0.5) + \minimumNodeSize/2 + \minimumNodeSize*0}, {-0*\verticalSpacing}) {5};
\node[treeNode] (n-1-0) at ( {(\treeDiagramWidth-(\minimumNodeSize*2))/(2)*(0+0.5) + \minimumNodeSize/2 + \minimumNodeSize*0}, {-1*\verticalSpacing}) {4};
\draw[treeEdge] (n-0-0) -- (n-1-0);
\node[treeNode] (n-1-1) at ( {(\treeDiagramWidth-(\minimumNodeSize*2))/(2)*(1+0.5) + \minimumNodeSize/2 + \minimumNodeSize*1}, {-1*\verticalSpacing}) {7};
\node[treeNode] (n-2-2) at ( {(\treeDiagramWidth-(\minimumNodeSize*4))/(4)*(2+0.5) + \minimumNodeSize/2 + \minimumNodeSize*2}, {-2*\verticalSpacing}) {6};
\draw[treeEdge] (n-1-1) -- (n-2-2);
\node[treeNode] (n-2-3) at ( {(\treeDiagramWidth-(\minimumNodeSize*4))/(4)*(3+0.5) + \minimumNodeSize/2 + \minimumNodeSize*3}, {-2*\verticalSpacing}) {8};
\node[treeNode] (n-3-7) at ( {(\treeDiagramWidth-(\minimumNodeSize*8))/(8)*(7+0.5) + \minimumNodeSize/2 + \minimumNodeSize*7}, {-3*\verticalSpacing}) {9};
\draw[treeEdge] (n-2-3) -- (n-3-7);
\draw[treeEdge] (n-1-1) -- (n-2-3);
\draw[treeEdge] (n-0-0) -- (n-1-1);
\end{tikzpicture}\end{center}
\endgroup\par




\par\begingroup
\newcommand{\treeDiagramWidth}{0.9\textwidth}
\newcommand{\minimumNodeSize}{8mm}
\newcommand{\verticalSpacing}{10mm}
\begin{center}\begin{tikzpicture}
\tikzstyle{treeNode}=[minimum width=\minimumNodeSize,minimum height=\minimumNodeSize,circle,draw,inner sep=1mm]
\tikzstyle{treeEdge}=[->]
\node[minimum width=\treeDiagramWidth,rectangle] at (\treeDiagramWidth/2,0) {};\node[treeNode] (n-0-0) at ( {(\treeDiagramWidth-(\minimumNodeSize*1))/(1)*(0+0.5) + \minimumNodeSize/2 + \minimumNodeSize*0}, {-0*\verticalSpacing}) {7};
\node[treeNode] (n-1-0) at ( {(\treeDiagramWidth-(\minimumNodeSize*2))/(2)*(0+0.5) + \minimumNodeSize/2 + \minimumNodeSize*0}, {-1*\verticalSpacing}) {5};
\node[treeNode] (n-2-0) at ( {(\treeDiagramWidth-(\minimumNodeSize*4))/(4)*(0+0.5) + \minimumNodeSize/2 + \minimumNodeSize*0}, {-2*\verticalSpacing}) {4};
\draw[treeEdge] (n-1-0) -- (n-2-0);
\draw[treeEdge] (n-0-0) -- (n-1-0);
\node[treeNode] (n-1-1) at ( {(\treeDiagramWidth-(\minimumNodeSize*2))/(2)*(1+0.5) + \minimumNodeSize/2 + \minimumNodeSize*1}, {-1*\verticalSpacing}) {8};
\node[treeNode] (n-2-2) at ( {(\treeDiagramWidth-(\minimumNodeSize*4))/(4)*(2+0.5) + \minimumNodeSize/2 + \minimumNodeSize*2}, {-2*\verticalSpacing}) {6};
\draw[treeEdge] (n-1-1) -- (n-2-2);
\node[treeNode] (n-2-3) at ( {(\treeDiagramWidth-(\minimumNodeSize*4))/(4)*(3+0.5) + \minimumNodeSize/2 + \minimumNodeSize*3}, {-2*\verticalSpacing}) {9};
\draw[treeEdge] (n-1-1) -- (n-2-3);
\draw[treeEdge] (n-0-0) -- (n-1-1);
\end{tikzpicture}\end{center}
\endgroup\par
 % This is a lot like a #include in C++: it brings in the contents of problem1.1.tex and puts it here.

\noindent \textbf{Problem 2.} Show the right rotation of the subtree rooted at 27.  Be sure to specify the X, Y, and Z subtrees used in the rotation.

\par\begingroup
\newcommand{\treeDiagramWidth}{0.9\textwidth}
\newcommand{\minimumNodeSize}{8mm}
\newcommand{\verticalSpacing}{10mm}
\begin{center}\begin{tikzpicture}
\tikzstyle{treeNode}=[minimum width=\minimumNodeSize,minimum height=\minimumNodeSize,circle,draw,inner sep=1mm]
\tikzstyle{treeEdge}=[->]
\node[minimum width=\treeDiagramWidth,rectangle] at (\treeDiagramWidth/2,0) {};\node[treeNode] (n-0-0) at ( {(\treeDiagramWidth-(\minimumNodeSize*1))/(1)*(0+0.5) + \minimumNodeSize/2 + \minimumNodeSize*0}, {-0*\verticalSpacing}) {19};
\node[treeNode] (n-1-0) at ( {(\treeDiagramWidth-(\minimumNodeSize*2))/(2)*(0+0.5) + \minimumNodeSize/2 + \minimumNodeSize*0}, {-1*\verticalSpacing}) {12};
\node[treeNode] (n-2-0) at ( {(\treeDiagramWidth-(\minimumNodeSize*4))/(4)*(0+0.5) + \minimumNodeSize/2 + \minimumNodeSize*0}, {-2*\verticalSpacing}) {8};
\node[treeNode] (n-3-0) at ( {(\treeDiagramWidth-(\minimumNodeSize*8))/(8)*(0+0.5) + \minimumNodeSize/2 + \minimumNodeSize*0}, {-3*\verticalSpacing}) {6};
\draw[treeEdge] (n-2-0) -- (n-3-0);
\draw[treeEdge] (n-1-0) -- (n-2-0);
\node[treeNode] (n-2-1) at ( {(\treeDiagramWidth-(\minimumNodeSize*4))/(4)*(1+0.5) + \minimumNodeSize/2 + \minimumNodeSize*1}, {-2*\verticalSpacing}) {13};
\node[treeNode] (n-3-3) at ( {(\treeDiagramWidth-(\minimumNodeSize*8))/(8)*(3+0.5) + \minimumNodeSize/2 + \minimumNodeSize*3}, {-3*\verticalSpacing}) {17};
\draw[treeEdge] (n-2-1) -- (n-3-3);
\draw[treeEdge] (n-1-0) -- (n-2-1);
\draw[treeEdge] (n-0-0) -- (n-1-0);
\node[treeNode] (n-1-1) at ( {(\treeDiagramWidth-(\minimumNodeSize*2))/(2)*(1+0.5) + \minimumNodeSize/2 + \minimumNodeSize*1}, {-1*\verticalSpacing}) {27};
\node[treeNode] (n-2-2) at ( {(\treeDiagramWidth-(\minimumNodeSize*4))/(4)*(2+0.5) + \minimumNodeSize/2 + \minimumNodeSize*2}, {-2*\verticalSpacing}) {22};
\node[treeNode] (n-3-4) at ( {(\treeDiagramWidth-(\minimumNodeSize*8))/(8)*(4+0.5) + \minimumNodeSize/2 + \minimumNodeSize*4}, {-3*\verticalSpacing}) {21};
\draw[treeEdge] (n-2-2) -- (n-3-4);
\node[treeNode] (n-3-5) at ( {(\treeDiagramWidth-(\minimumNodeSize*8))/(8)*(5+0.5) + \minimumNodeSize/2 + \minimumNodeSize*5}, {-3*\verticalSpacing}) {24};
\draw[treeEdge] (n-2-2) -- (n-3-5);
\draw[treeEdge] (n-1-1) -- (n-2-2);
\node[treeNode] (n-2-3) at ( {(\treeDiagramWidth-(\minimumNodeSize*4))/(4)*(3+0.5) + \minimumNodeSize/2 + \minimumNodeSize*3}, {-2*\verticalSpacing}) {30};
\draw[treeEdge] (n-1-1) -- (n-2-3);
\draw[treeEdge] (n-0-0) -- (n-1-1);
\end{tikzpicture}\end{center}
\endgroup\par




\par\begingroup
\newcommand{\treeDiagramWidth}{0.9\textwidth}
\newcommand{\minimumNodeSize}{8mm}
\newcommand{\verticalSpacing}{10mm}
\begin{center}\begin{tikzpicture}
\tikzstyle{treeNode}=[minimum width=\minimumNodeSize,minimum height=\minimumNodeSize,circle,draw,inner sep=1mm]
\tikzstyle{treeEdge}=[->]
\node[minimum width=\treeDiagramWidth,rectangle] at (\treeDiagramWidth/2,0) {};\node[treeNode] (n-0-0) at ( {(\treeDiagramWidth-(\minimumNodeSize*1))/(1)*(0+0.5) + \minimumNodeSize/2 + \minimumNodeSize*0}, {-0*\verticalSpacing}) {19};
\node[treeNode] (n-1-0) at ( {(\treeDiagramWidth-(\minimumNodeSize*2))/(2)*(0+0.5) + \minimumNodeSize/2 + \minimumNodeSize*0}, {-1*\verticalSpacing}) {12};
\node[treeNode] (n-2-0) at ( {(\treeDiagramWidth-(\minimumNodeSize*4))/(4)*(0+0.5) + \minimumNodeSize/2 + \minimumNodeSize*0}, {-2*\verticalSpacing}) {8};
\node[treeNode] (n-3-0) at ( {(\treeDiagramWidth-(\minimumNodeSize*8))/(8)*(0+0.5) + \minimumNodeSize/2 + \minimumNodeSize*0}, {-3*\verticalSpacing}) {6};
\draw[treeEdge] (n-2-0) -- (n-3-0);
\draw[treeEdge] (n-1-0) -- (n-2-0);
\node[treeNode] (n-2-1) at ( {(\treeDiagramWidth-(\minimumNodeSize*4))/(4)*(1+0.5) + \minimumNodeSize/2 + \minimumNodeSize*1}, {-2*\verticalSpacing}) {13};
\node[treeNode] (n-3-3) at ( {(\treeDiagramWidth-(\minimumNodeSize*8))/(8)*(3+0.5) + \minimumNodeSize/2 + \minimumNodeSize*3}, {-3*\verticalSpacing}) {17};
\draw[treeEdge] (n-2-1) -- (n-3-3);
\draw[treeEdge] (n-1-0) -- (n-2-1);
\draw[treeEdge] (n-0-0) -- (n-1-0);
\node[treeNode] (n-1-1) at ( {(\treeDiagramWidth-(\minimumNodeSize*2))/(2)*(1+0.5) + \minimumNodeSize/2 + \minimumNodeSize*1}, {-1*\verticalSpacing}) {22};
\node[treeNode] (n-2-2) at ( {(\treeDiagramWidth-(\minimumNodeSize*4))/(4)*(2+0.5) + \minimumNodeSize/2 + \minimumNodeSize*2}, {-2*\verticalSpacing}) {21};
\draw[treeEdge] (n-1-1) -- (n-2-2);
\node[treeNode] (n-2-3) at ( {(\treeDiagramWidth-(\minimumNodeSize*4))/(4)*(3+0.5) + \minimumNodeSize/2 + \minimumNodeSize*3}, {-2*\verticalSpacing}) {27};
\node[treeNode] (n-3-6) at ( {(\treeDiagramWidth-(\minimumNodeSize*8))/(8)*(6+0.5) + \minimumNodeSize/2 + \minimumNodeSize*6}, {-3*\verticalSpacing}) {24};
\draw[treeEdge] (n-2-3) -- (n-3-6);
\node[treeNode] (n-3-7) at ( {(\treeDiagramWidth-(\minimumNodeSize*8))/(8)*(7+0.5) + \minimumNodeSize/2 + \minimumNodeSize*7}, {-3*\verticalSpacing}) {30};
\draw[treeEdge] (n-2-3) -- (n-3-7);
\draw[treeEdge] (n-1-1) -- (n-2-3);
\draw[treeEdge] (n-0-0) -- (n-1-1);
\end{tikzpicture}\end{center}
\endgroup\par


\noindent \textbf{Problem 3.} Using the appropriate AVL tree algorithm, insert the value 12 into the following tree.  Show the tree before and after rebalancing.

\par\begingroup
\newcommand{\treeDiagramWidth}{0.9\textwidth}
\newcommand{\minimumNodeSize}{8mm}
\newcommand{\verticalSpacing}{10mm}
\begin{center}\begin{tikzpicture}
\tikzstyle{treeNode}=[minimum width=\minimumNodeSize,minimum height=\minimumNodeSize,circle,draw,inner sep=1mm]
\tikzstyle{treeEdge}=[->]
\node[minimum width=\treeDiagramWidth,rectangle] at (\treeDiagramWidth/2,0) {};\node[treeNode] (n-0-0) at ( {(\treeDiagramWidth-(\minimumNodeSize*1))/(1)*(0+0.5) + \minimumNodeSize/2 + \minimumNodeSize*0}, {-0*\verticalSpacing}) {30};
\node[treeNode] (n-1-0) at ( {(\treeDiagramWidth-(\minimumNodeSize*2))/(2)*(0+0.5) + \minimumNodeSize/2 + \minimumNodeSize*0}, {-1*\verticalSpacing}) {20};
\node[treeNode] (n-2-0) at ( {(\treeDiagramWidth-(\minimumNodeSize*4))/(4)*(0+0.5) + \minimumNodeSize/2 + \minimumNodeSize*0}, {-2*\verticalSpacing}) {10};
\draw[treeEdge] (n-1-0) -- (n-2-0);
\draw[treeEdge] (n-0-0) -- (n-1-0);
\node[treeNode] (n-1-1) at ( {(\treeDiagramWidth-(\minimumNodeSize*2))/(2)*(1+0.5) + \minimumNodeSize/2 + \minimumNodeSize*1}, {-1*\verticalSpacing}) {40};
\draw[treeEdge] (n-0-0) -- (n-1-1);
\end{tikzpicture}\end{center}
\endgroup\par

Since 12 is smaller than the root, 30, it goes to the left
12 becomes leaf node


\par\begingroup
\newcommand{\treeDiagramWidth}{0.9\textwidth}
\newcommand{\minimumNodeSize}{8mm}
\newcommand{\verticalSpacing}{10mm}
\begin{center}\begin{tikzpicture}
\tikzstyle{treeNode}=[minimum width=\minimumNodeSize,minimum height=\minimumNodeSize,circle,draw,inner sep=1mm]
\tikzstyle{treeEdge}=[->]
\node[minimum width=\treeDiagramWidth,rectangle] at (\treeDiagramWidth/2,0) {};\node[treeNode] (n-0-0) at ( {(\treeDiagramWidth-(\minimumNodeSize*1))/(1)*(0+0.5) + \minimumNodeSize/2 + \minimumNodeSize*0}, {-0*\verticalSpacing}) {30};
\node[treeNode] (n-1-0) at ( {(\treeDiagramWidth-(\minimumNodeSize*2))/(2)*(0+0.5) + \minimumNodeSize/2 + \minimumNodeSize*0}, {-1*\verticalSpacing}) {20};
\node[treeNode] (n-2-0) at ( {(\treeDiagramWidth-(\minimumNodeSize*4))/(4)*(0+0.5) + \minimumNodeSize/2 + \minimumNodeSize*0}, {-2*\verticalSpacing}) {10};
\node[treeNode] (n-3-1) at ( {(\treeDiagramWidth-(\minimumNodeSize*8))/(8)*(1+0.5) + \minimumNodeSize/2 + \minimumNodeSize*1}, {-3*\verticalSpacing}) {12};
\node[treeNode] (n-4-3) at ( {(\treeDiagramWidth-(\minimumNodeSize*16))/(16)*(3+0.5) + \minimumNodeSize/2 + \minimumNodeSize*3}, {-4*\verticalSpacing}) {30};
\node[treeNode] (n-5-6) at ( {(\treeDiagramWidth-(\minimumNodeSize*32))/(32)*(6+0.5) + \minimumNodeSize/2 + \minimumNodeSize*6}, {-5*\verticalSpacing}) {10};
\node[treeNode] (n-6-13) at ( {(\treeDiagramWidth-(\minimumNodeSize*64))/(64)*(13+0.5) + \minimumNodeSize/2 + \minimumNodeSize*13}, {-6*\verticalSpacing}) {20};
\node[treeNode] (n-7-26) at ( {(\treeDiagramWidth-(\minimumNodeSize*128))/(128)*(26+0.5) + \minimumNodeSize/2 + \minimumNodeSize*26}, {-7*\verticalSpacing}) {12};
\draw[treeEdge] (n-6-13) -- (n-7-26);
\draw[treeEdge] (n-5-6) -- (n-6-13);
\draw[treeEdge] (n-4-3) -- (n-5-6);
\node[treeNode] (n-5-7) at ( {(\treeDiagramWidth-(\minimumNodeSize*32))/(32)*(7+0.5) + \minimumNodeSize/2 + \minimumNodeSize*7}, {-5*\verticalSpacing}) {40};
\draw[treeEdge] (n-4-3) -- (n-5-7);
\draw[treeEdge] (n-3-1) -- (n-4-3);
\draw[treeEdge] (n-2-0) -- (n-3-1);
\draw[treeEdge] (n-1-0) -- (n-2-0);
\draw[treeEdge] (n-0-0) -- (n-1-0);
\node[treeNode] (n-1-1) at ( {(\treeDiagramWidth-(\minimumNodeSize*2))/(2)*(1+0.5) + \minimumNodeSize/2 + \minimumNodeSize*1}, {-1*\verticalSpacing}) {40};
\draw[treeEdge] (n-0-0) -- (n-1-1);
\end{tikzpicture}\end{center}
\endgroup\par

Since it's inbalanced and LSTH - RSTH > 1
right rotation is performed


\par\begingroup
\newcommand{\treeDiagramWidth}{0.9\textwidth}
\newcommand{\minimumNodeSize}{8mm}
\newcommand{\verticalSpacing}{10mm}
\begin{center}\begin{tikzpicture}
\tikzstyle{treeNode}=[minimum width=\minimumNodeSize,minimum height=\minimumNodeSize,circle,draw,inner sep=1mm]
\tikzstyle{treeEdge}=[->]
\node[minimum width=\treeDiagramWidth,rectangle] at (\treeDiagramWidth/2,0) {};\node[treeNode] (n-0-0) at ( {(\treeDiagramWidth-(\minimumNodeSize*1))/(1)*(0+0.5) + \minimumNodeSize/2 + \minimumNodeSize*0}, {-0*\verticalSpacing}) {20};
\node[treeNode] (n-1-0) at ( {(\treeDiagramWidth-(\minimumNodeSize*2))/(2)*(0+0.5) + \minimumNodeSize/2 + \minimumNodeSize*0}, {-1*\verticalSpacing}) {10};
\node[treeNode] (n-2-1) at ( {(\treeDiagramWidth-(\minimumNodeSize*4))/(4)*(1+0.5) + \minimumNodeSize/2 + \minimumNodeSize*1}, {-2*\verticalSpacing}) {12};
\node[treeNode] (n-3-3) at ( {(\treeDiagramWidth-(\minimumNodeSize*8))/(8)*(3+0.5) + \minimumNodeSize/2 + \minimumNodeSize*3}, {-3*\verticalSpacing}) {30};
\node[treeNode] (n-4-6) at ( {(\treeDiagramWidth-(\minimumNodeSize*16))/(16)*(6+0.5) + \minimumNodeSize/2 + \minimumNodeSize*6}, {-4*\verticalSpacing}) {12};
\node[treeNode] (n-5-12) at ( {(\treeDiagramWidth-(\minimumNodeSize*32))/(32)*(12+0.5) + \minimumNodeSize/2 + \minimumNodeSize*12}, {-5*\verticalSpacing}) {10};
\draw[treeEdge] (n-4-6) -- (n-5-12);
\node[treeNode] (n-5-13) at ( {(\treeDiagramWidth-(\minimumNodeSize*32))/(32)*(13+0.5) + \minimumNodeSize/2 + \minimumNodeSize*13}, {-5*\verticalSpacing}) {20};
\draw[treeEdge] (n-4-6) -- (n-5-13);
\draw[treeEdge] (n-3-3) -- (n-4-6);
\node[treeNode] (n-4-7) at ( {(\treeDiagramWidth-(\minimumNodeSize*16))/(16)*(7+0.5) + \minimumNodeSize/2 + \minimumNodeSize*7}, {-4*\verticalSpacing}) {40};
\draw[treeEdge] (n-3-3) -- (n-4-7);
\draw[treeEdge] (n-2-1) -- (n-3-3);
\draw[treeEdge] (n-1-0) -- (n-2-1);
\draw[treeEdge] (n-0-0) -- (n-1-0);
\node[treeNode] (n-1-1) at ( {(\treeDiagramWidth-(\minimumNodeSize*2))/(2)*(1+0.5) + \minimumNodeSize/2 + \minimumNodeSize*1}, {-1*\verticalSpacing}) {30};
\node[treeNode] (n-2-3) at ( {(\treeDiagramWidth-(\minimumNodeSize*4))/(4)*(3+0.5) + \minimumNodeSize/2 + \minimumNodeSize*3}, {-2*\verticalSpacing}) {40};
\draw[treeEdge] (n-1-1) -- (n-2-3);
\draw[treeEdge] (n-0-0) -- (n-1-1);
\end{tikzpicture}\end{center}
\endgroup\par


\noindent \textbf{Problem 4.} Using the appropriate AVL tree algorithm, remove the value 54 from the following tree.  Show the tree before and after rebalancing.

\par\begingroup
\newcommand{\treeDiagramWidth}{0.9\textwidth}
\newcommand{\minimumNodeSize}{8mm}
\newcommand{\verticalSpacing}{10mm}
\begin{center}\begin{tikzpicture}
\tikzstyle{treeNode}=[minimum width=\minimumNodeSize,minimum height=\minimumNodeSize,circle,draw,inner sep=1mm]
\tikzstyle{treeEdge}=[->]
\node[minimum width=\treeDiagramWidth,rectangle] at (\treeDiagramWidth/2,0) {};\node[treeNode] (n-0-0) at ( {(\treeDiagramWidth-(\minimumNodeSize*1))/(1)*(0+0.5) + \minimumNodeSize/2 + \minimumNodeSize*0}, {-0*\verticalSpacing}) {40};
\node[treeNode] (n-1-0) at ( {(\treeDiagramWidth-(\minimumNodeSize*2))/(2)*(0+0.5) + \minimumNodeSize/2 + \minimumNodeSize*0}, {-1*\verticalSpacing}) {24};
\node[treeNode] (n-2-0) at ( {(\treeDiagramWidth-(\minimumNodeSize*4))/(4)*(0+0.5) + \minimumNodeSize/2 + \minimumNodeSize*0}, {-2*\verticalSpacing}) {12};
\node[treeNode] (n-3-0) at ( {(\treeDiagramWidth-(\minimumNodeSize*8))/(8)*(0+0.5) + \minimumNodeSize/2 + \minimumNodeSize*0}, {-3*\verticalSpacing}) {11};
\draw[treeEdge] (n-2-0) -- (n-3-0);
\draw[treeEdge] (n-1-0) -- (n-2-0);
\node[treeNode] (n-2-1) at ( {(\treeDiagramWidth-(\minimumNodeSize*4))/(4)*(1+0.5) + \minimumNodeSize/2 + \minimumNodeSize*1}, {-2*\verticalSpacing}) {36};
\draw[treeEdge] (n-1-0) -- (n-2-1);
\draw[treeEdge] (n-0-0) -- (n-1-0);
\node[treeNode] (n-1-1) at ( {(\treeDiagramWidth-(\minimumNodeSize*2))/(2)*(1+0.5) + \minimumNodeSize/2 + \minimumNodeSize*1}, {-1*\verticalSpacing}) {54};
\node[treeNode] (n-2-2) at ( {(\treeDiagramWidth-(\minimumNodeSize*4))/(4)*(2+0.5) + \minimumNodeSize/2 + \minimumNodeSize*2}, {-2*\verticalSpacing}) {46};
\node[treeNode] (n-3-4) at ( {(\treeDiagramWidth-(\minimumNodeSize*8))/(8)*(4+0.5) + \minimumNodeSize/2 + \minimumNodeSize*4}, {-3*\verticalSpacing}) {42};
\node[treeNode] (n-4-9) at ( {(\treeDiagramWidth-(\minimumNodeSize*16))/(16)*(9+0.5) + \minimumNodeSize/2 + \minimumNodeSize*9}, {-4*\verticalSpacing}) {43};
\draw[treeEdge] (n-3-4) -- (n-4-9);
\draw[treeEdge] (n-2-2) -- (n-3-4);
\node[treeNode] (n-3-5) at ( {(\treeDiagramWidth-(\minimumNodeSize*8))/(8)*(5+0.5) + \minimumNodeSize/2 + \minimumNodeSize*5}, {-3*\verticalSpacing}) {50};
\draw[treeEdge] (n-2-2) -- (n-3-5);
\draw[treeEdge] (n-1-1) -- (n-2-2);
\node[treeNode] (n-2-3) at ( {(\treeDiagramWidth-(\minimumNodeSize*4))/(4)*(3+0.5) + \minimumNodeSize/2 + \minimumNodeSize*3}, {-2*\verticalSpacing}) {80};
\node[treeNode] (n-3-6) at ( {(\treeDiagramWidth-(\minimumNodeSize*8))/(8)*(6+0.5) + \minimumNodeSize/2 + \minimumNodeSize*6}, {-3*\verticalSpacing}) {79};
\draw[treeEdge] (n-2-3) -- (n-3-6);
\node[treeNode] (n-3-7) at ( {(\treeDiagramWidth-(\minimumNodeSize*8))/(8)*(7+0.5) + \minimumNodeSize/2 + \minimumNodeSize*7}, {-3*\verticalSpacing}) {81};
\draw[treeEdge] (n-2-3) -- (n-3-7);
\draw[treeEdge] (n-1-1) -- (n-2-3);
\draw[treeEdge] (n-0-0) -- (n-1-1);
\end{tikzpicture}\end{center}
\endgroup\par




\par\begingroup
\newcommand{\treeDiagramWidth}{0.9\textwidth}
\newcommand{\minimumNodeSize}{8mm}
\newcommand{\verticalSpacing}{10mm}
\begin{center}\begin{tikzpicture}
\tikzstyle{treeNode}=[minimum width=\minimumNodeSize,minimum height=\minimumNodeSize,circle,draw,inner sep=1mm]
\tikzstyle{treeEdge}=[->]
\node[minimum width=\treeDiagramWidth,rectangle] at (\treeDiagramWidth/2,0) {};\node[treeNode] (n-0-0) at ( {(\treeDiagramWidth-(\minimumNodeSize*1))/(1)*(0+0.5) + \minimumNodeSize/2 + \minimumNodeSize*0}, {-0*\verticalSpacing}) {40};
\node[treeNode] (n-1-0) at ( {(\treeDiagramWidth-(\minimumNodeSize*2))/(2)*(0+0.5) + \minimumNodeSize/2 + \minimumNodeSize*0}, {-1*\verticalSpacing}) {24};
\node[treeNode] (n-2-0) at ( {(\treeDiagramWidth-(\minimumNodeSize*4))/(4)*(0+0.5) + \minimumNodeSize/2 + \minimumNodeSize*0}, {-2*\verticalSpacing}) {12};
\node[treeNode] (n-3-0) at ( {(\treeDiagramWidth-(\minimumNodeSize*8))/(8)*(0+0.5) + \minimumNodeSize/2 + \minimumNodeSize*0}, {-3*\verticalSpacing}) {11};
\draw[treeEdge] (n-2-0) -- (n-3-0);
\draw[treeEdge] (n-1-0) -- (n-2-0);
\node[treeNode] (n-2-1) at ( {(\treeDiagramWidth-(\minimumNodeSize*4))/(4)*(1+0.5) + \minimumNodeSize/2 + \minimumNodeSize*1}, {-2*\verticalSpacing}) {36};
\draw[treeEdge] (n-1-0) -- (n-2-1);
\draw[treeEdge] (n-0-0) -- (n-1-0);
\node[treeNode] (n-1-1) at ( {(\treeDiagramWidth-(\minimumNodeSize*2))/(2)*(1+0.5) + \minimumNodeSize/2 + \minimumNodeSize*1}, {-1*\verticalSpacing}) {79};
\node[treeNode] (n-2-2) at ( {(\treeDiagramWidth-(\minimumNodeSize*4))/(4)*(2+0.5) + \minimumNodeSize/2 + \minimumNodeSize*2}, {-2*\verticalSpacing}) {46};
\node[treeNode] (n-3-4) at ( {(\treeDiagramWidth-(\minimumNodeSize*8))/(8)*(4+0.5) + \minimumNodeSize/2 + \minimumNodeSize*4}, {-3*\verticalSpacing}) {42};
\node[treeNode] (n-4-9) at ( {(\treeDiagramWidth-(\minimumNodeSize*16))/(16)*(9+0.5) + \minimumNodeSize/2 + \minimumNodeSize*9}, {-4*\verticalSpacing}) {43};
\draw[treeEdge] (n-3-4) -- (n-4-9);
\draw[treeEdge] (n-2-2) -- (n-3-4);
\node[treeNode] (n-3-5) at ( {(\treeDiagramWidth-(\minimumNodeSize*8))/(8)*(5+0.5) + \minimumNodeSize/2 + \minimumNodeSize*5}, {-3*\verticalSpacing}) {50};
\draw[treeEdge] (n-2-2) -- (n-3-5);
\draw[treeEdge] (n-1-1) -- (n-2-2);
\node[treeNode] (n-2-3) at ( {(\treeDiagramWidth-(\minimumNodeSize*4))/(4)*(3+0.5) + \minimumNodeSize/2 + \minimumNodeSize*3}, {-2*\verticalSpacing}) {80};
\node[treeNode] (n-3-7) at ( {(\treeDiagramWidth-(\minimumNodeSize*8))/(8)*(7+0.5) + \minimumNodeSize/2 + \minimumNodeSize*7}, {-3*\verticalSpacing}) {81};
\draw[treeEdge] (n-2-3) -- (n-3-7);
\draw[treeEdge] (n-1-1) -- (n-2-3);
\draw[treeEdge] (n-0-0) -- (n-1-1);
\end{tikzpicture}\end{center}
\endgroup\par


\section{Heaps}

\noindent \textbf{Problem 1.} Show the addition of the element 9 to the max-heap below.  First, show the addition of 9 to the tree; then, show each bubbling step.

\par\begingroup
\newcommand{\treeDiagramWidth}{0.9\textwidth}
\newcommand{\minimumNodeSize}{8mm}
\newcommand{\verticalSpacing}{10mm}
\begin{center}\begin{tikzpicture}
\tikzstyle{treeNode}=[minimum width=\minimumNodeSize,minimum height=\minimumNodeSize,circle,draw,inner sep=1mm]
\tikzstyle{treeEdge}=[->]
\node[minimum width=\treeDiagramWidth,rectangle] at (\treeDiagramWidth/2,0) {};\node[treeNode] (n-0-0) at ( {(\treeDiagramWidth-(\minimumNodeSize*1))/(1)*(0+0.5) + \minimumNodeSize/2 + \minimumNodeSize*0}, {-0*\verticalSpacing}) {9};
\node[treeNode] (n-1-0) at ( {(\treeDiagramWidth-(\minimumNodeSize*2))/(2)*(0+0.5) + \minimumNodeSize/2 + \minimumNodeSize*0}, {-1*\verticalSpacing}) {7};
\node[treeNode] (n-2-0) at ( {(\treeDiagramWidth-(\minimumNodeSize*4))/(4)*(0+0.5) + \minimumNodeSize/2 + \minimumNodeSize*0}, {-2*\verticalSpacing}) {4};
\node[treeNode] (n-3-0) at ( {(\treeDiagramWidth-(\minimumNodeSize*8))/(8)*(0+0.5) + \minimumNodeSize/2 + \minimumNodeSize*0}, {-3*\verticalSpacing}) {3};
\draw[treeEdge] (n-2-0) -- (n-3-0);
\node[treeNode] (n-3-1) at ( {(\treeDiagramWidth-(\minimumNodeSize*8))/(8)*(1+0.5) + \minimumNodeSize/2 + \minimumNodeSize*1}, {-3*\verticalSpacing}) {2};
\draw[treeEdge] (n-2-0) -- (n-3-1);
\draw[treeEdge] (n-1-0) -- (n-2-0);
\node[treeNode] (n-2-1) at ( {(\treeDiagramWidth-(\minimumNodeSize*4))/(4)*(1+0.5) + \minimumNodeSize/2 + \minimumNodeSize*1}, {-2*\verticalSpacing}) {5};
\node[treeNode] (n-3-2) at ( {(\treeDiagramWidth-(\minimumNodeSize*8))/(8)*(2+0.5) + \minimumNodeSize/2 + \minimumNodeSize*2}, {-3*\verticalSpacing}) {1};
\draw[treeEdge] (n-2-1) -- (n-3-2);
\draw[treeEdge] (n-1-0) -- (n-2-1);
\draw[treeEdge] (n-0-0) -- (n-1-0);
\node[treeNode] (n-1-1) at ( {(\treeDiagramWidth-(\minimumNodeSize*2))/(2)*(1+0.5) + \minimumNodeSize/2 + \minimumNodeSize*1}, {-1*\verticalSpacing}) {5};
\node[treeNode] (n-2-2) at ( {(\treeDiagramWidth-(\minimumNodeSize*4))/(4)*(2+0.5) + \minimumNodeSize/2 + \minimumNodeSize*2}, {-2*\verticalSpacing}) {3};
\draw[treeEdge] (n-1-1) -- (n-2-2);
\node[treeNode] (n-2-3) at ( {(\treeDiagramWidth-(\minimumNodeSize*4))/(4)*(3+0.5) + \minimumNodeSize/2 + \minimumNodeSize*3}, {-2*\verticalSpacing}) {2};
\draw[treeEdge] (n-1-1) -- (n-2-3);
\draw[treeEdge] (n-0-0) -- (n-1-1);
\end{tikzpicture}\end{center}
\endgroup\par

Since the BT has to remain complete, 9 will be inserted
as the right child of the left 5.


\par\begingroup
\newcommand{\treeDiagramWidth}{0.9\textwidth}
\newcommand{\minimumNodeSize}{8mm}
\newcommand{\verticalSpacing}{10mm}
\begin{center}\begin{tikzpicture}
\tikzstyle{treeNode}=[minimum width=\minimumNodeSize,minimum height=\minimumNodeSize,circle,draw,inner sep=1mm]
\tikzstyle{treeEdge}=[->]
\node[minimum width=\treeDiagramWidth,rectangle] at (\treeDiagramWidth/2,0) {};\node[treeNode] (n-0-0) at ( {(\treeDiagramWidth-(\minimumNodeSize*1))/(1)*(0+0.5) + \minimumNodeSize/2 + \minimumNodeSize*0}, {-0*\verticalSpacing}) {9};
\node[treeNode] (n-1-0) at ( {(\treeDiagramWidth-(\minimumNodeSize*2))/(2)*(0+0.5) + \minimumNodeSize/2 + \minimumNodeSize*0}, {-1*\verticalSpacing}) {7};
\node[treeNode] (n-2-0) at ( {(\treeDiagramWidth-(\minimumNodeSize*4))/(4)*(0+0.5) + \minimumNodeSize/2 + \minimumNodeSize*0}, {-2*\verticalSpacing}) {4};
\node[treeNode] (n-3-0) at ( {(\treeDiagramWidth-(\minimumNodeSize*8))/(8)*(0+0.5) + \minimumNodeSize/2 + \minimumNodeSize*0}, {-3*\verticalSpacing}) {3};
\draw[treeEdge] (n-2-0) -- (n-3-0);
\node[treeNode] (n-3-1) at ( {(\treeDiagramWidth-(\minimumNodeSize*8))/(8)*(1+0.5) + \minimumNodeSize/2 + \minimumNodeSize*1}, {-3*\verticalSpacing}) {2};
\draw[treeEdge] (n-2-0) -- (n-3-1);
\draw[treeEdge] (n-1-0) -- (n-2-0);
\node[treeNode] (n-2-1) at ( {(\treeDiagramWidth-(\minimumNodeSize*4))/(4)*(1+0.5) + \minimumNodeSize/2 + \minimumNodeSize*1}, {-2*\verticalSpacing}) {5};
\node[treeNode] (n-3-2) at ( {(\treeDiagramWidth-(\minimumNodeSize*8))/(8)*(2+0.5) + \minimumNodeSize/2 + \minimumNodeSize*2}, {-3*\verticalSpacing}) {1};
\draw[treeEdge] (n-2-1) -- (n-3-2);
\node[treeNode] (n-3-3) at ( {(\treeDiagramWidth-(\minimumNodeSize*8))/(8)*(3+0.5) + \minimumNodeSize/2 + \minimumNodeSize*3}, {-3*\verticalSpacing}) {9};
\draw[treeEdge] (n-2-1) -- (n-3-3);
\draw[treeEdge] (n-1-0) -- (n-2-1);
\draw[treeEdge] (n-0-0) -- (n-1-0);
\node[treeNode] (n-1-1) at ( {(\treeDiagramWidth-(\minimumNodeSize*2))/(2)*(1+0.5) + \minimumNodeSize/2 + \minimumNodeSize*1}, {-1*\verticalSpacing}) {5};
\node[treeNode] (n-2-2) at ( {(\treeDiagramWidth-(\minimumNodeSize*4))/(4)*(2+0.5) + \minimumNodeSize/2 + \minimumNodeSize*2}, {-2*\verticalSpacing}) {3};
\draw[treeEdge] (n-1-1) -- (n-2-2);
\node[treeNode] (n-2-3) at ( {(\treeDiagramWidth-(\minimumNodeSize*4))/(4)*(3+0.5) + \minimumNodeSize/2 + \minimumNodeSize*3}, {-2*\verticalSpacing}) {2};
\draw[treeEdge] (n-1-1) -- (n-2-3);
\draw[treeEdge] (n-0-0) -- (n-1-1);
\end{tikzpicture}\end{center}
\endgroup\par

The max-heap invariant must be met, so 9 is swapped with 5
since 9 > 5.


\par\begingroup
\newcommand{\treeDiagramWidth}{0.9\textwidth}
\newcommand{\minimumNodeSize}{8mm}
\newcommand{\verticalSpacing}{10mm}
\begin{center}\begin{tikzpicture}
\tikzstyle{treeNode}=[minimum width=\minimumNodeSize,minimum height=\minimumNodeSize,circle,draw,inner sep=1mm]
\tikzstyle{treeEdge}=[->]
\node[minimum width=\treeDiagramWidth,rectangle] at (\treeDiagramWidth/2,0) {};\node[treeNode] (n-0-0) at ( {(\treeDiagramWidth-(\minimumNodeSize*1))/(1)*(0+0.5) + \minimumNodeSize/2 + \minimumNodeSize*0}, {-0*\verticalSpacing}) {9};
\node[treeNode] (n-1-0) at ( {(\treeDiagramWidth-(\minimumNodeSize*2))/(2)*(0+0.5) + \minimumNodeSize/2 + \minimumNodeSize*0}, {-1*\verticalSpacing}) {7};
\node[treeNode] (n-2-0) at ( {(\treeDiagramWidth-(\minimumNodeSize*4))/(4)*(0+0.5) + \minimumNodeSize/2 + \minimumNodeSize*0}, {-2*\verticalSpacing}) {4};
\node[treeNode] (n-3-0) at ( {(\treeDiagramWidth-(\minimumNodeSize*8))/(8)*(0+0.5) + \minimumNodeSize/2 + \minimumNodeSize*0}, {-3*\verticalSpacing}) {3};
\draw[treeEdge] (n-2-0) -- (n-3-0);
\node[treeNode] (n-3-1) at ( {(\treeDiagramWidth-(\minimumNodeSize*8))/(8)*(1+0.5) + \minimumNodeSize/2 + \minimumNodeSize*1}, {-3*\verticalSpacing}) {2};
\draw[treeEdge] (n-2-0) -- (n-3-1);
\draw[treeEdge] (n-1-0) -- (n-2-0);
\node[treeNode] (n-2-1) at ( {(\treeDiagramWidth-(\minimumNodeSize*4))/(4)*(1+0.5) + \minimumNodeSize/2 + \minimumNodeSize*1}, {-2*\verticalSpacing}) {9};
\node[treeNode] (n-3-2) at ( {(\treeDiagramWidth-(\minimumNodeSize*8))/(8)*(2+0.5) + \minimumNodeSize/2 + \minimumNodeSize*2}, {-3*\verticalSpacing}) {1};
\draw[treeEdge] (n-2-1) -- (n-3-2);
\node[treeNode] (n-3-3) at ( {(\treeDiagramWidth-(\minimumNodeSize*8))/(8)*(3+0.5) + \minimumNodeSize/2 + \minimumNodeSize*3}, {-3*\verticalSpacing}) {5};
\draw[treeEdge] (n-2-1) -- (n-3-3);
\draw[treeEdge] (n-1-0) -- (n-2-1);
\draw[treeEdge] (n-0-0) -- (n-1-0);
\node[treeNode] (n-1-1) at ( {(\treeDiagramWidth-(\minimumNodeSize*2))/(2)*(1+0.5) + \minimumNodeSize/2 + \minimumNodeSize*1}, {-1*\verticalSpacing}) {5};
\node[treeNode] (n-2-2) at ( {(\treeDiagramWidth-(\minimumNodeSize*4))/(4)*(2+0.5) + \minimumNodeSize/2 + \minimumNodeSize*2}, {-2*\verticalSpacing}) {3};
\draw[treeEdge] (n-1-1) -- (n-2-2);
\node[treeNode] (n-2-3) at ( {(\treeDiagramWidth-(\minimumNodeSize*4))/(4)*(3+0.5) + \minimumNodeSize/2 + \minimumNodeSize*3}, {-2*\verticalSpacing}) {2};
\draw[treeEdge] (n-1-1) -- (n-2-3);
\draw[treeEdge] (n-0-0) -- (n-1-1);
\end{tikzpicture}\end{center}
\endgroup\par

Since 9 is now the child of 7, it must also swap.
This leaves 9 to be a child of the root 9, which meets the heap invariant.


\par\begingroup
\newcommand{\treeDiagramWidth}{0.9\textwidth}
\newcommand{\minimumNodeSize}{8mm}
\newcommand{\verticalSpacing}{10mm}
\begin{center}\begin{tikzpicture}
\tikzstyle{treeNode}=[minimum width=\minimumNodeSize,minimum height=\minimumNodeSize,circle,draw,inner sep=1mm]
\tikzstyle{treeEdge}=[->]
\node[minimum width=\treeDiagramWidth,rectangle] at (\treeDiagramWidth/2,0) {};\node[treeNode] (n-0-0) at ( {(\treeDiagramWidth-(\minimumNodeSize*1))/(1)*(0+0.5) + \minimumNodeSize/2 + \minimumNodeSize*0}, {-0*\verticalSpacing}) {9};
\node[treeNode] (n-1-0) at ( {(\treeDiagramWidth-(\minimumNodeSize*2))/(2)*(0+0.5) + \minimumNodeSize/2 + \minimumNodeSize*0}, {-1*\verticalSpacing}) {9};
\node[treeNode] (n-2-0) at ( {(\treeDiagramWidth-(\minimumNodeSize*4))/(4)*(0+0.5) + \minimumNodeSize/2 + \minimumNodeSize*0}, {-2*\verticalSpacing}) {4};
\node[treeNode] (n-3-0) at ( {(\treeDiagramWidth-(\minimumNodeSize*8))/(8)*(0+0.5) + \minimumNodeSize/2 + \minimumNodeSize*0}, {-3*\verticalSpacing}) {3};
\draw[treeEdge] (n-2-0) -- (n-3-0);
\node[treeNode] (n-3-1) at ( {(\treeDiagramWidth-(\minimumNodeSize*8))/(8)*(1+0.5) + \minimumNodeSize/2 + \minimumNodeSize*1}, {-3*\verticalSpacing}) {2};
\draw[treeEdge] (n-2-0) -- (n-3-1);
\draw[treeEdge] (n-1-0) -- (n-2-0);
\node[treeNode] (n-2-1) at ( {(\treeDiagramWidth-(\minimumNodeSize*4))/(4)*(1+0.5) + \minimumNodeSize/2 + \minimumNodeSize*1}, {-2*\verticalSpacing}) {7};
\node[treeNode] (n-3-2) at ( {(\treeDiagramWidth-(\minimumNodeSize*8))/(8)*(2+0.5) + \minimumNodeSize/2 + \minimumNodeSize*2}, {-3*\verticalSpacing}) {1};
\draw[treeEdge] (n-2-1) -- (n-3-2);
\node[treeNode] (n-3-3) at ( {(\treeDiagramWidth-(\minimumNodeSize*8))/(8)*(3+0.5) + \minimumNodeSize/2 + \minimumNodeSize*3}, {-3*\verticalSpacing}) {5};
\draw[treeEdge] (n-2-1) -- (n-3-3);
\draw[treeEdge] (n-1-0) -- (n-2-1);
\draw[treeEdge] (n-0-0) -- (n-1-0);
\node[treeNode] (n-1-1) at ( {(\treeDiagramWidth-(\minimumNodeSize*2))/(2)*(1+0.5) + \minimumNodeSize/2 + \minimumNodeSize*1}, {-1*\verticalSpacing}) {5};
\node[treeNode] (n-2-2) at ( {(\treeDiagramWidth-(\minimumNodeSize*4))/(4)*(2+0.5) + \minimumNodeSize/2 + \minimumNodeSize*2}, {-2*\verticalSpacing}) {3};
\draw[treeEdge] (n-1-1) -- (n-2-2);
\node[treeNode] (n-2-3) at ( {(\treeDiagramWidth-(\minimumNodeSize*4))/(4)*(3+0.5) + \minimumNodeSize/2 + \minimumNodeSize*3}, {-2*\verticalSpacing}) {2};
\draw[treeEdge] (n-1-1) -- (n-2-3);
\draw[treeEdge] (n-0-0) -- (n-1-1);
\end{tikzpicture}\end{center}
\endgroup\par


\noindent \textbf{Problem 2.} Show the removal of the top element of this max-heap.  First, show the swap of the root node; then, show each bubbling step.

\par\begingroup
\newcommand{\treeDiagramWidth}{0.9\textwidth}
\newcommand{\minimumNodeSize}{8mm}
\newcommand{\verticalSpacing}{10mm}
\begin{center}\begin{tikzpicture}
\tikzstyle{treeNode}=[minimum width=\minimumNodeSize,minimum height=\minimumNodeSize,circle,draw,inner sep=1mm]
\tikzstyle{treeEdge}=[->]
\node[minimum width=\treeDiagramWidth,rectangle] at (\treeDiagramWidth/2,0) {};\node[treeNode] (n-0-0) at ( {(\treeDiagramWidth-(\minimumNodeSize*1))/(1)*(0+0.5) + \minimumNodeSize/2 + \minimumNodeSize*0}, {-0*\verticalSpacing}) {9};
\node[treeNode] (n-1-0) at ( {(\treeDiagramWidth-(\minimumNodeSize*2))/(2)*(0+0.5) + \minimumNodeSize/2 + \minimumNodeSize*0}, {-1*\verticalSpacing}) {4};
\node[treeNode] (n-2-0) at ( {(\treeDiagramWidth-(\minimumNodeSize*4))/(4)*(0+0.5) + \minimumNodeSize/2 + \minimumNodeSize*0}, {-2*\verticalSpacing}) {4};
\node[treeNode] (n-3-0) at ( {(\treeDiagramWidth-(\minimumNodeSize*8))/(8)*(0+0.5) + \minimumNodeSize/2 + \minimumNodeSize*0}, {-3*\verticalSpacing}) {3};
\draw[treeEdge] (n-2-0) -- (n-3-0);
\node[treeNode] (n-3-1) at ( {(\treeDiagramWidth-(\minimumNodeSize*8))/(8)*(1+0.5) + \minimumNodeSize/2 + \minimumNodeSize*1}, {-3*\verticalSpacing}) {1};
\draw[treeEdge] (n-2-0) -- (n-3-1);
\draw[treeEdge] (n-1-0) -- (n-2-0);
\node[treeNode] (n-2-1) at ( {(\treeDiagramWidth-(\minimumNodeSize*4))/(4)*(1+0.5) + \minimumNodeSize/2 + \minimumNodeSize*1}, {-2*\verticalSpacing}) {3};
\node[treeNode] (n-3-2) at ( {(\treeDiagramWidth-(\minimumNodeSize*8))/(8)*(2+0.5) + \minimumNodeSize/2 + \minimumNodeSize*2}, {-3*\verticalSpacing}) {2};
\draw[treeEdge] (n-2-1) -- (n-3-2);
\node[treeNode] (n-3-3) at ( {(\treeDiagramWidth-(\minimumNodeSize*8))/(8)*(3+0.5) + \minimumNodeSize/2 + \minimumNodeSize*3}, {-3*\verticalSpacing}) {2};
\draw[treeEdge] (n-2-1) -- (n-3-3);
\draw[treeEdge] (n-1-0) -- (n-2-1);
\draw[treeEdge] (n-0-0) -- (n-1-0);
\node[treeNode] (n-1-1) at ( {(\treeDiagramWidth-(\minimumNodeSize*2))/(2)*(1+0.5) + \minimumNodeSize/2 + \minimumNodeSize*1}, {-1*\verticalSpacing}) {8};
\node[treeNode] (n-2-2) at ( {(\treeDiagramWidth-(\minimumNodeSize*4))/(4)*(2+0.5) + \minimumNodeSize/2 + \minimumNodeSize*2}, {-2*\verticalSpacing}) {7};
\node[treeNode] (n-3-4) at ( {(\treeDiagramWidth-(\minimumNodeSize*8))/(8)*(4+0.5) + \minimumNodeSize/2 + \minimumNodeSize*4}, {-3*\verticalSpacing}) {6};
\draw[treeEdge] (n-2-2) -- (n-3-4);
\node[treeNode] (n-3-5) at ( {(\treeDiagramWidth-(\minimumNodeSize*8))/(8)*(5+0.5) + \minimumNodeSize/2 + \minimumNodeSize*5}, {-3*\verticalSpacing}) {4};
\draw[treeEdge] (n-2-2) -- (n-3-5);
\draw[treeEdge] (n-1-1) -- (n-2-2);
\node[treeNode] (n-2-3) at ( {(\treeDiagramWidth-(\minimumNodeSize*4))/(4)*(3+0.5) + \minimumNodeSize/2 + \minimumNodeSize*3}, {-2*\verticalSpacing}) {2};
\node[treeNode] (n-3-6) at ( {(\treeDiagramWidth-(\minimumNodeSize*8))/(8)*(6+0.5) + \minimumNodeSize/2 + \minimumNodeSize*6}, {-3*\verticalSpacing}) {1};
\draw[treeEdge] (n-2-3) -- (n-3-6);
\draw[treeEdge] (n-1-1) -- (n-2-3);
\draw[treeEdge] (n-0-0) -- (n-1-1);
\end{tikzpicture}\end{center}
\endgroup\par

Since trying to remove 9, it will be swapped with last node, which is 1


\par\begingroup
\newcommand{\treeDiagramWidth}{0.9\textwidth}
\newcommand{\minimumNodeSize}{8mm}
\newcommand{\verticalSpacing}{10mm}
\begin{center}\begin{tikzpicture}
\tikzstyle{treeNode}=[minimum width=\minimumNodeSize,minimum height=\minimumNodeSize,circle,draw,inner sep=1mm]
\tikzstyle{treeEdge}=[->]
\node[minimum width=\treeDiagramWidth,rectangle] at (\treeDiagramWidth/2,0) {};\node[treeNode] (n-0-0) at ( {(\treeDiagramWidth-(\minimumNodeSize*1))/(1)*(0+0.5) + \minimumNodeSize/2 + \minimumNodeSize*0}, {-0*\verticalSpacing}) {1};
\node[treeNode] (n-1-0) at ( {(\treeDiagramWidth-(\minimumNodeSize*2))/(2)*(0+0.5) + \minimumNodeSize/2 + \minimumNodeSize*0}, {-1*\verticalSpacing}) {4};
\node[treeNode] (n-2-0) at ( {(\treeDiagramWidth-(\minimumNodeSize*4))/(4)*(0+0.5) + \minimumNodeSize/2 + \minimumNodeSize*0}, {-2*\verticalSpacing}) {4};
\node[treeNode] (n-3-0) at ( {(\treeDiagramWidth-(\minimumNodeSize*8))/(8)*(0+0.5) + \minimumNodeSize/2 + \minimumNodeSize*0}, {-3*\verticalSpacing}) {3};
\draw[treeEdge] (n-2-0) -- (n-3-0);
\node[treeNode] (n-3-1) at ( {(\treeDiagramWidth-(\minimumNodeSize*8))/(8)*(1+0.5) + \minimumNodeSize/2 + \minimumNodeSize*1}, {-3*\verticalSpacing}) {1};
\draw[treeEdge] (n-2-0) -- (n-3-1);
\draw[treeEdge] (n-1-0) -- (n-2-0);
\node[treeNode] (n-2-1) at ( {(\treeDiagramWidth-(\minimumNodeSize*4))/(4)*(1+0.5) + \minimumNodeSize/2 + \minimumNodeSize*1}, {-2*\verticalSpacing}) {3};
\node[treeNode] (n-3-2) at ( {(\treeDiagramWidth-(\minimumNodeSize*8))/(8)*(2+0.5) + \minimumNodeSize/2 + \minimumNodeSize*2}, {-3*\verticalSpacing}) {2};
\draw[treeEdge] (n-2-1) -- (n-3-2);
\node[treeNode] (n-3-3) at ( {(\treeDiagramWidth-(\minimumNodeSize*8))/(8)*(3+0.5) + \minimumNodeSize/2 + \minimumNodeSize*3}, {-3*\verticalSpacing}) {2};
\draw[treeEdge] (n-2-1) -- (n-3-3);
\draw[treeEdge] (n-1-0) -- (n-2-1);
\draw[treeEdge] (n-0-0) -- (n-1-0);
\node[treeNode] (n-1-1) at ( {(\treeDiagramWidth-(\minimumNodeSize*2))/(2)*(1+0.5) + \minimumNodeSize/2 + \minimumNodeSize*1}, {-1*\verticalSpacing}) {8};
\node[treeNode] (n-2-2) at ( {(\treeDiagramWidth-(\minimumNodeSize*4))/(4)*(2+0.5) + \minimumNodeSize/2 + \minimumNodeSize*2}, {-2*\verticalSpacing}) {7};
\node[treeNode] (n-3-4) at ( {(\treeDiagramWidth-(\minimumNodeSize*8))/(8)*(4+0.5) + \minimumNodeSize/2 + \minimumNodeSize*4}, {-3*\verticalSpacing}) {6};
\draw[treeEdge] (n-2-2) -- (n-3-4);
\node[treeNode] (n-3-5) at ( {(\treeDiagramWidth-(\minimumNodeSize*8))/(8)*(5+0.5) + \minimumNodeSize/2 + \minimumNodeSize*5}, {-3*\verticalSpacing}) {4};
\draw[treeEdge] (n-2-2) -- (n-3-5);
\draw[treeEdge] (n-1-1) -- (n-2-2);
\node[treeNode] (n-2-3) at ( {(\treeDiagramWidth-(\minimumNodeSize*4))/(4)*(3+0.5) + \minimumNodeSize/2 + \minimumNodeSize*3}, {-2*\verticalSpacing}) {2};
\node[treeNode] (n-3-6) at ( {(\treeDiagramWidth-(\minimumNodeSize*8))/(8)*(6+0.5) + \minimumNodeSize/2 + \minimumNodeSize*6}, {-3*\verticalSpacing}) {9};
\draw[treeEdge] (n-2-3) -- (n-3-6);
\draw[treeEdge] (n-1-1) -- (n-2-3);
\draw[treeEdge] (n-0-0) -- (n-1-1);
\end{tikzpicture}\end{center}
\endgroup\par

Once swapped, 9 can be removed.


\par\begingroup
\newcommand{\treeDiagramWidth}{0.9\textwidth}
\newcommand{\minimumNodeSize}{8mm}
\newcommand{\verticalSpacing}{10mm}
\begin{center}\begin{tikzpicture}
\tikzstyle{treeNode}=[minimum width=\minimumNodeSize,minimum height=\minimumNodeSize,circle,draw,inner sep=1mm]
\tikzstyle{treeEdge}=[->]
\node[minimum width=\treeDiagramWidth,rectangle] at (\treeDiagramWidth/2,0) {};\node[treeNode] (n-0-0) at ( {(\treeDiagramWidth-(\minimumNodeSize*1))/(1)*(0+0.5) + \minimumNodeSize/2 + \minimumNodeSize*0}, {-0*\verticalSpacing}) {1};
\node[treeNode] (n-1-0) at ( {(\treeDiagramWidth-(\minimumNodeSize*2))/(2)*(0+0.5) + \minimumNodeSize/2 + \minimumNodeSize*0}, {-1*\verticalSpacing}) {4};
\node[treeNode] (n-2-0) at ( {(\treeDiagramWidth-(\minimumNodeSize*4))/(4)*(0+0.5) + \minimumNodeSize/2 + \minimumNodeSize*0}, {-2*\verticalSpacing}) {4};
\node[treeNode] (n-3-0) at ( {(\treeDiagramWidth-(\minimumNodeSize*8))/(8)*(0+0.5) + \minimumNodeSize/2 + \minimumNodeSize*0}, {-3*\verticalSpacing}) {3};
\draw[treeEdge] (n-2-0) -- (n-3-0);
\node[treeNode] (n-3-1) at ( {(\treeDiagramWidth-(\minimumNodeSize*8))/(8)*(1+0.5) + \minimumNodeSize/2 + \minimumNodeSize*1}, {-3*\verticalSpacing}) {1};
\draw[treeEdge] (n-2-0) -- (n-3-1);
\draw[treeEdge] (n-1-0) -- (n-2-0);
\node[treeNode] (n-2-1) at ( {(\treeDiagramWidth-(\minimumNodeSize*4))/(4)*(1+0.5) + \minimumNodeSize/2 + \minimumNodeSize*1}, {-2*\verticalSpacing}) {3};
\node[treeNode] (n-3-2) at ( {(\treeDiagramWidth-(\minimumNodeSize*8))/(8)*(2+0.5) + \minimumNodeSize/2 + \minimumNodeSize*2}, {-3*\verticalSpacing}) {2};
\draw[treeEdge] (n-2-1) -- (n-3-2);
\node[treeNode] (n-3-3) at ( {(\treeDiagramWidth-(\minimumNodeSize*8))/(8)*(3+0.5) + \minimumNodeSize/2 + \minimumNodeSize*3}, {-3*\verticalSpacing}) {2};
\draw[treeEdge] (n-2-1) -- (n-3-3);
\draw[treeEdge] (n-1-0) -- (n-2-1);
\draw[treeEdge] (n-0-0) -- (n-1-0);
\node[treeNode] (n-1-1) at ( {(\treeDiagramWidth-(\minimumNodeSize*2))/(2)*(1+0.5) + \minimumNodeSize/2 + \minimumNodeSize*1}, {-1*\verticalSpacing}) {8};
\node[treeNode] (n-2-2) at ( {(\treeDiagramWidth-(\minimumNodeSize*4))/(4)*(2+0.5) + \minimumNodeSize/2 + \minimumNodeSize*2}, {-2*\verticalSpacing}) {7};
\node[treeNode] (n-3-4) at ( {(\treeDiagramWidth-(\minimumNodeSize*8))/(8)*(4+0.5) + \minimumNodeSize/2 + \minimumNodeSize*4}, {-3*\verticalSpacing}) {6};
\draw[treeEdge] (n-2-2) -- (n-3-4);
\node[treeNode] (n-3-5) at ( {(\treeDiagramWidth-(\minimumNodeSize*8))/(8)*(5+0.5) + \minimumNodeSize/2 + \minimumNodeSize*5}, {-3*\verticalSpacing}) {4};
\draw[treeEdge] (n-2-2) -- (n-3-5);
\draw[treeEdge] (n-1-1) -- (n-2-2);
\node[treeNode] (n-2-3) at ( {(\treeDiagramWidth-(\minimumNodeSize*4))/(4)*(3+0.5) + \minimumNodeSize/2 + \minimumNodeSize*3}, {-2*\verticalSpacing}) {2};
\draw[treeEdge] (n-1-1) -- (n-2-3);
\draw[treeEdge] (n-0-0) -- (n-1-1);
\end{tikzpicture}\end{center}
\endgroup\par

Although 9 was removed, 1 is now the root. It will have to be swapped with 8 since list[left]<list[right]


\par\begingroup
\newcommand{\treeDiagramWidth}{0.9\textwidth}
\newcommand{\minimumNodeSize}{8mm}
\newcommand{\verticalSpacing}{10mm}
\begin{center}\begin{tikzpicture}
\tikzstyle{treeNode}=[minimum width=\minimumNodeSize,minimum height=\minimumNodeSize,circle,draw,inner sep=1mm]
\tikzstyle{treeEdge}=[->]
\node[minimum width=\treeDiagramWidth,rectangle] at (\treeDiagramWidth/2,0) {};\node[treeNode] (n-0-0) at ( {(\treeDiagramWidth-(\minimumNodeSize*1))/(1)*(0+0.5) + \minimumNodeSize/2 + \minimumNodeSize*0}, {-0*\verticalSpacing}) {8};
\node[treeNode] (n-1-0) at ( {(\treeDiagramWidth-(\minimumNodeSize*2))/(2)*(0+0.5) + \minimumNodeSize/2 + \minimumNodeSize*0}, {-1*\verticalSpacing}) {4};
\node[treeNode] (n-2-0) at ( {(\treeDiagramWidth-(\minimumNodeSize*4))/(4)*(0+0.5) + \minimumNodeSize/2 + \minimumNodeSize*0}, {-2*\verticalSpacing}) {4};
\node[treeNode] (n-3-0) at ( {(\treeDiagramWidth-(\minimumNodeSize*8))/(8)*(0+0.5) + \minimumNodeSize/2 + \minimumNodeSize*0}, {-3*\verticalSpacing}) {3};
\draw[treeEdge] (n-2-0) -- (n-3-0);
\node[treeNode] (n-3-1) at ( {(\treeDiagramWidth-(\minimumNodeSize*8))/(8)*(1+0.5) + \minimumNodeSize/2 + \minimumNodeSize*1}, {-3*\verticalSpacing}) {1};
\draw[treeEdge] (n-2-0) -- (n-3-1);
\draw[treeEdge] (n-1-0) -- (n-2-0);
\node[treeNode] (n-2-1) at ( {(\treeDiagramWidth-(\minimumNodeSize*4))/(4)*(1+0.5) + \minimumNodeSize/2 + \minimumNodeSize*1}, {-2*\verticalSpacing}) {3};
\node[treeNode] (n-3-2) at ( {(\treeDiagramWidth-(\minimumNodeSize*8))/(8)*(2+0.5) + \minimumNodeSize/2 + \minimumNodeSize*2}, {-3*\verticalSpacing}) {2};
\draw[treeEdge] (n-2-1) -- (n-3-2);
\node[treeNode] (n-3-3) at ( {(\treeDiagramWidth-(\minimumNodeSize*8))/(8)*(3+0.5) + \minimumNodeSize/2 + \minimumNodeSize*3}, {-3*\verticalSpacing}) {2};
\draw[treeEdge] (n-2-1) -- (n-3-3);
\draw[treeEdge] (n-1-0) -- (n-2-1);
\draw[treeEdge] (n-0-0) -- (n-1-0);
\node[treeNode] (n-1-1) at ( {(\treeDiagramWidth-(\minimumNodeSize*2))/(2)*(1+0.5) + \minimumNodeSize/2 + \minimumNodeSize*1}, {-1*\verticalSpacing}) {1};
\node[treeNode] (n-2-2) at ( {(\treeDiagramWidth-(\minimumNodeSize*4))/(4)*(2+0.5) + \minimumNodeSize/2 + \minimumNodeSize*2}, {-2*\verticalSpacing}) {7};
\node[treeNode] (n-3-4) at ( {(\treeDiagramWidth-(\minimumNodeSize*8))/(8)*(4+0.5) + \minimumNodeSize/2 + \minimumNodeSize*4}, {-3*\verticalSpacing}) {6};
\draw[treeEdge] (n-2-2) -- (n-3-4);
\node[treeNode] (n-3-5) at ( {(\treeDiagramWidth-(\minimumNodeSize*8))/(8)*(5+0.5) + \minimumNodeSize/2 + \minimumNodeSize*5}, {-3*\verticalSpacing}) {4};
\draw[treeEdge] (n-2-2) -- (n-3-5);
\draw[treeEdge] (n-1-1) -- (n-2-2);
\node[treeNode] (n-2-3) at ( {(\treeDiagramWidth-(\minimumNodeSize*4))/(4)*(3+0.5) + \minimumNodeSize/2 + \minimumNodeSize*3}, {-2*\verticalSpacing}) {2};
\draw[treeEdge] (n-1-1) -- (n-2-3);
\draw[treeEdge] (n-0-0) -- (n-1-1);
\end{tikzpicture}\end{center}
\endgroup\par

This continues. 1 is swapped with 7 since 7 > 2.


\par\begingroup
\newcommand{\treeDiagramWidth}{0.9\textwidth}
\newcommand{\minimumNodeSize}{8mm}
\newcommand{\verticalSpacing}{10mm}
\begin{center}\begin{tikzpicture}
\tikzstyle{treeNode}=[minimum width=\minimumNodeSize,minimum height=\minimumNodeSize,circle,draw,inner sep=1mm]
\tikzstyle{treeEdge}=[->]
\node[minimum width=\treeDiagramWidth,rectangle] at (\treeDiagramWidth/2,0) {};\node[treeNode] (n-0-0) at ( {(\treeDiagramWidth-(\minimumNodeSize*1))/(1)*(0+0.5) + \minimumNodeSize/2 + \minimumNodeSize*0}, {-0*\verticalSpacing}) {8};
\node[treeNode] (n-1-0) at ( {(\treeDiagramWidth-(\minimumNodeSize*2))/(2)*(0+0.5) + \minimumNodeSize/2 + \minimumNodeSize*0}, {-1*\verticalSpacing}) {4};
\node[treeNode] (n-2-0) at ( {(\treeDiagramWidth-(\minimumNodeSize*4))/(4)*(0+0.5) + \minimumNodeSize/2 + \minimumNodeSize*0}, {-2*\verticalSpacing}) {4};
\node[treeNode] (n-3-0) at ( {(\treeDiagramWidth-(\minimumNodeSize*8))/(8)*(0+0.5) + \minimumNodeSize/2 + \minimumNodeSize*0}, {-3*\verticalSpacing}) {3};
\draw[treeEdge] (n-2-0) -- (n-3-0);
\node[treeNode] (n-3-1) at ( {(\treeDiagramWidth-(\minimumNodeSize*8))/(8)*(1+0.5) + \minimumNodeSize/2 + \minimumNodeSize*1}, {-3*\verticalSpacing}) {1};
\draw[treeEdge] (n-2-0) -- (n-3-1);
\draw[treeEdge] (n-1-0) -- (n-2-0);
\node[treeNode] (n-2-1) at ( {(\treeDiagramWidth-(\minimumNodeSize*4))/(4)*(1+0.5) + \minimumNodeSize/2 + \minimumNodeSize*1}, {-2*\verticalSpacing}) {3};
\node[treeNode] (n-3-2) at ( {(\treeDiagramWidth-(\minimumNodeSize*8))/(8)*(2+0.5) + \minimumNodeSize/2 + \minimumNodeSize*2}, {-3*\verticalSpacing}) {2};
\draw[treeEdge] (n-2-1) -- (n-3-2);
\node[treeNode] (n-3-3) at ( {(\treeDiagramWidth-(\minimumNodeSize*8))/(8)*(3+0.5) + \minimumNodeSize/2 + \minimumNodeSize*3}, {-3*\verticalSpacing}) {2};
\draw[treeEdge] (n-2-1) -- (n-3-3);
\draw[treeEdge] (n-1-0) -- (n-2-1);
\draw[treeEdge] (n-0-0) -- (n-1-0);
\node[treeNode] (n-1-1) at ( {(\treeDiagramWidth-(\minimumNodeSize*2))/(2)*(1+0.5) + \minimumNodeSize/2 + \minimumNodeSize*1}, {-1*\verticalSpacing}) {7};
\node[treeNode] (n-2-2) at ( {(\treeDiagramWidth-(\minimumNodeSize*4))/(4)*(2+0.5) + \minimumNodeSize/2 + \minimumNodeSize*2}, {-2*\verticalSpacing}) {1};
\node[treeNode] (n-3-4) at ( {(\treeDiagramWidth-(\minimumNodeSize*8))/(8)*(4+0.5) + \minimumNodeSize/2 + \minimumNodeSize*4}, {-3*\verticalSpacing}) {6};
\draw[treeEdge] (n-2-2) -- (n-3-4);
\node[treeNode] (n-3-5) at ( {(\treeDiagramWidth-(\minimumNodeSize*8))/(8)*(5+0.5) + \minimumNodeSize/2 + \minimumNodeSize*5}, {-3*\verticalSpacing}) {4};
\draw[treeEdge] (n-2-2) -- (n-3-5);
\draw[treeEdge] (n-1-1) -- (n-2-2);
\node[treeNode] (n-2-3) at ( {(\treeDiagramWidth-(\minimumNodeSize*4))/(4)*(3+0.5) + \minimumNodeSize/2 + \minimumNodeSize*3}, {-2*\verticalSpacing}) {2};
\draw[treeEdge] (n-1-1) -- (n-2-3);
\draw[treeEdge] (n-0-0) -- (n-1-1);
\end{tikzpicture}\end{center}
\endgroup\par

Like before, 1 is swapped with list[left] since list[left]>list[right]


\par\begingroup
\newcommand{\treeDiagramWidth}{0.9\textwidth}
\newcommand{\minimumNodeSize}{8mm}
\newcommand{\verticalSpacing}{10mm}
\begin{center}\begin{tikzpicture}
\tikzstyle{treeNode}=[minimum width=\minimumNodeSize,minimum height=\minimumNodeSize,circle,draw,inner sep=1mm]
\tikzstyle{treeEdge}=[->]
\node[minimum width=\treeDiagramWidth,rectangle] at (\treeDiagramWidth/2,0) {};\node[treeNode] (n-0-0) at ( {(\treeDiagramWidth-(\minimumNodeSize*1))/(1)*(0+0.5) + \minimumNodeSize/2 + \minimumNodeSize*0}, {-0*\verticalSpacing}) {8};
\node[treeNode] (n-1-0) at ( {(\treeDiagramWidth-(\minimumNodeSize*2))/(2)*(0+0.5) + \minimumNodeSize/2 + \minimumNodeSize*0}, {-1*\verticalSpacing}) {4};
\node[treeNode] (n-2-0) at ( {(\treeDiagramWidth-(\minimumNodeSize*4))/(4)*(0+0.5) + \minimumNodeSize/2 + \minimumNodeSize*0}, {-2*\verticalSpacing}) {4};
\node[treeNode] (n-3-0) at ( {(\treeDiagramWidth-(\minimumNodeSize*8))/(8)*(0+0.5) + \minimumNodeSize/2 + \minimumNodeSize*0}, {-3*\verticalSpacing}) {3};
\draw[treeEdge] (n-2-0) -- (n-3-0);
\node[treeNode] (n-3-1) at ( {(\treeDiagramWidth-(\minimumNodeSize*8))/(8)*(1+0.5) + \minimumNodeSize/2 + \minimumNodeSize*1}, {-3*\verticalSpacing}) {1};
\draw[treeEdge] (n-2-0) -- (n-3-1);
\draw[treeEdge] (n-1-0) -- (n-2-0);
\node[treeNode] (n-2-1) at ( {(\treeDiagramWidth-(\minimumNodeSize*4))/(4)*(1+0.5) + \minimumNodeSize/2 + \minimumNodeSize*1}, {-2*\verticalSpacing}) {3};
\node[treeNode] (n-3-2) at ( {(\treeDiagramWidth-(\minimumNodeSize*8))/(8)*(2+0.5) + \minimumNodeSize/2 + \minimumNodeSize*2}, {-3*\verticalSpacing}) {2};
\draw[treeEdge] (n-2-1) -- (n-3-2);
\node[treeNode] (n-3-3) at ( {(\treeDiagramWidth-(\minimumNodeSize*8))/(8)*(3+0.5) + \minimumNodeSize/2 + \minimumNodeSize*3}, {-3*\verticalSpacing}) {2};
\draw[treeEdge] (n-2-1) -- (n-3-3);
\draw[treeEdge] (n-1-0) -- (n-2-1);
\draw[treeEdge] (n-0-0) -- (n-1-0);
\node[treeNode] (n-1-1) at ( {(\treeDiagramWidth-(\minimumNodeSize*2))/(2)*(1+0.5) + \minimumNodeSize/2 + \minimumNodeSize*1}, {-1*\verticalSpacing}) {7};
\node[treeNode] (n-2-2) at ( {(\treeDiagramWidth-(\minimumNodeSize*4))/(4)*(2+0.5) + \minimumNodeSize/2 + \minimumNodeSize*2}, {-2*\verticalSpacing}) {6};
\node[treeNode] (n-3-4) at ( {(\treeDiagramWidth-(\minimumNodeSize*8))/(8)*(4+0.5) + \minimumNodeSize/2 + \minimumNodeSize*4}, {-3*\verticalSpacing}) {1};
\draw[treeEdge] (n-2-2) -- (n-3-4);
\node[treeNode] (n-3-5) at ( {(\treeDiagramWidth-(\minimumNodeSize*8))/(8)*(5+0.5) + \minimumNodeSize/2 + \minimumNodeSize*5}, {-3*\verticalSpacing}) {4};
\draw[treeEdge] (n-2-2) -- (n-3-5);
\draw[treeEdge] (n-1-1) -- (n-2-2);
\node[treeNode] (n-2-3) at ( {(\treeDiagramWidth-(\minimumNodeSize*4))/(4)*(3+0.5) + \minimumNodeSize/2 + \minimumNodeSize*3}, {-2*\verticalSpacing}) {2};
\draw[treeEdge] (n-1-1) -- (n-2-3);
\draw[treeEdge] (n-0-0) -- (n-1-1);
\end{tikzpicture}\end{center}
\endgroup\par


\noindent \textbf{Problem 3.} Consider the sequence of elements \texttt{[5,4,2,3,2,8,5]}.  Using the representation discussed in class, show the tree to which this sequence corresponds.  Then, show the \textit{heapification} of this tree; that is, show how this tree is transformed into a heap.  Demonstrate each bubbling step.

\par\begingroup
\newcommand{\treeDiagramWidth}{0.9\textwidth}
\newcommand{\minimumNodeSize}{8mm}
\newcommand{\verticalSpacing}{10mm}
\begin{center}\begin{tikzpicture}
\tikzstyle{treeNode}=[minimum width=\minimumNodeSize,minimum height=\minimumNodeSize,circle,draw,inner sep=1mm]
\tikzstyle{treeEdge}=[->]
\node[minimum width=\treeDiagramWidth,rectangle] at (\treeDiagramWidth/2,0) {};\node[treeNode] (n-0-0) at ( {(\treeDiagramWidth-(\minimumNodeSize*1))/(1)*(0+0.5) + \minimumNodeSize/2 + \minimumNodeSize*0}, {-0*\verticalSpacing}) {5};
\node[treeNode] (n-1-0) at ( {(\treeDiagramWidth-(\minimumNodeSize*2))/(2)*(0+0.5) + \minimumNodeSize/2 + \minimumNodeSize*0}, {-1*\verticalSpacing}) {4};
\node[treeNode] (n-2-0) at ( {(\treeDiagramWidth-(\minimumNodeSize*4))/(4)*(0+0.5) + \minimumNodeSize/2 + \minimumNodeSize*0}, {-2*\verticalSpacing}) {3};
\draw[treeEdge] (n-1-0) -- (n-2-0);
\node[treeNode] (n-2-1) at ( {(\treeDiagramWidth-(\minimumNodeSize*4))/(4)*(1+0.5) + \minimumNodeSize/2 + \minimumNodeSize*1}, {-2*\verticalSpacing}) {2};
\draw[treeEdge] (n-1-0) -- (n-2-1);
\draw[treeEdge] (n-0-0) -- (n-1-0);
\node[treeNode] (n-1-1) at ( {(\treeDiagramWidth-(\minimumNodeSize*2))/(2)*(1+0.5) + \minimumNodeSize/2 + \minimumNodeSize*1}, {-1*\verticalSpacing}) {2};
\node[treeNode] (n-2-2) at ( {(\treeDiagramWidth-(\minimumNodeSize*4))/(4)*(2+0.5) + \minimumNodeSize/2 + \minimumNodeSize*2}, {-2*\verticalSpacing}) {8};
\draw[treeEdge] (n-1-1) -- (n-2-2);
\node[treeNode] (n-2-3) at ( {(\treeDiagramWidth-(\minimumNodeSize*4))/(4)*(3+0.5) + \minimumNodeSize/2 + \minimumNodeSize*3}, {-2*\verticalSpacing}) {5};
\draw[treeEdge] (n-1-1) -- (n-2-3);
\draw[treeEdge] (n-0-0) -- (n-1-1);
\end{tikzpicture}\end{center}
\endgroup\par

The 8 would first swap with 2, to make the subtree a heap.
Since the other subtree is already a heap, no swapping is needed.


\par\begingroup
\newcommand{\treeDiagramWidth}{0.9\textwidth}
\newcommand{\minimumNodeSize}{8mm}
\newcommand{\verticalSpacing}{10mm}
\begin{center}\begin{tikzpicture}
\tikzstyle{treeNode}=[minimum width=\minimumNodeSize,minimum height=\minimumNodeSize,circle,draw,inner sep=1mm]
\tikzstyle{treeEdge}=[->]
\node[minimum width=\treeDiagramWidth,rectangle] at (\treeDiagramWidth/2,0) {};\node[treeNode] (n-0-0) at ( {(\treeDiagramWidth-(\minimumNodeSize*1))/(1)*(0+0.5) + \minimumNodeSize/2 + \minimumNodeSize*0}, {-0*\verticalSpacing}) {5};
\node[treeNode] (n-1-0) at ( {(\treeDiagramWidth-(\minimumNodeSize*2))/(2)*(0+0.5) + \minimumNodeSize/2 + \minimumNodeSize*0}, {-1*\verticalSpacing}) {4};
\node[treeNode] (n-2-0) at ( {(\treeDiagramWidth-(\minimumNodeSize*4))/(4)*(0+0.5) + \minimumNodeSize/2 + \minimumNodeSize*0}, {-2*\verticalSpacing}) {3};
\draw[treeEdge] (n-1-0) -- (n-2-0);
\node[treeNode] (n-2-1) at ( {(\treeDiagramWidth-(\minimumNodeSize*4))/(4)*(1+0.5) + \minimumNodeSize/2 + \minimumNodeSize*1}, {-2*\verticalSpacing}) {2};
\draw[treeEdge] (n-1-0) -- (n-2-1);
\draw[treeEdge] (n-0-0) -- (n-1-0);
\node[treeNode] (n-1-1) at ( {(\treeDiagramWidth-(\minimumNodeSize*2))/(2)*(1+0.5) + \minimumNodeSize/2 + \minimumNodeSize*1}, {-1*\verticalSpacing}) {8};
\node[treeNode] (n-2-2) at ( {(\treeDiagramWidth-(\minimumNodeSize*4))/(4)*(2+0.5) + \minimumNodeSize/2 + \minimumNodeSize*2}, {-2*\verticalSpacing}) {2};
\draw[treeEdge] (n-1-1) -- (n-2-2);
\node[treeNode] (n-2-3) at ( {(\treeDiagramWidth-(\minimumNodeSize*4))/(4)*(3+0.5) + \minimumNodeSize/2 + \minimumNodeSize*3}, {-2*\verticalSpacing}) {5};
\draw[treeEdge] (n-1-1) -- (n-2-3);
\draw[treeEdge] (n-0-0) -- (n-1-1);
\end{tikzpicture}\end{center}
\endgroup\par

Since the subtrees are heaps, the root 5 can now be bubbled down.
8 > 4, so 5 is swapped with 8


\par\begingroup
\newcommand{\treeDiagramWidth}{0.9\textwidth}
\newcommand{\minimumNodeSize}{8mm}
\newcommand{\verticalSpacing}{10mm}
\begin{center}\begin{tikzpicture}
\tikzstyle{treeNode}=[minimum width=\minimumNodeSize,minimum height=\minimumNodeSize,circle,draw,inner sep=1mm]
\tikzstyle{treeEdge}=[->]
\node[minimum width=\treeDiagramWidth,rectangle] at (\treeDiagramWidth/2,0) {};\node[treeNode] (n-0-0) at ( {(\treeDiagramWidth-(\minimumNodeSize*1))/(1)*(0+0.5) + \minimumNodeSize/2 + \minimumNodeSize*0}, {-0*\verticalSpacing}) {8};
\node[treeNode] (n-1-0) at ( {(\treeDiagramWidth-(\minimumNodeSize*2))/(2)*(0+0.5) + \minimumNodeSize/2 + \minimumNodeSize*0}, {-1*\verticalSpacing}) {4};
\node[treeNode] (n-2-0) at ( {(\treeDiagramWidth-(\minimumNodeSize*4))/(4)*(0+0.5) + \minimumNodeSize/2 + \minimumNodeSize*0}, {-2*\verticalSpacing}) {3};
\draw[treeEdge] (n-1-0) -- (n-2-0);
\node[treeNode] (n-2-1) at ( {(\treeDiagramWidth-(\minimumNodeSize*4))/(4)*(1+0.5) + \minimumNodeSize/2 + \minimumNodeSize*1}, {-2*\verticalSpacing}) {2};
\draw[treeEdge] (n-1-0) -- (n-2-1);
\draw[treeEdge] (n-0-0) -- (n-1-0);
\node[treeNode] (n-1-1) at ( {(\treeDiagramWidth-(\minimumNodeSize*2))/(2)*(1+0.5) + \minimumNodeSize/2 + \minimumNodeSize*1}, {-1*\verticalSpacing}) {5};
\node[treeNode] (n-2-2) at ( {(\treeDiagramWidth-(\minimumNodeSize*4))/(4)*(2+0.5) + \minimumNodeSize/2 + \minimumNodeSize*2}, {-2*\verticalSpacing}) {2};
\draw[treeEdge] (n-1-1) -- (n-2-2);
\node[treeNode] (n-2-3) at ( {(\treeDiagramWidth-(\minimumNodeSize*4))/(4)*(3+0.5) + \minimumNodeSize/2 + \minimumNodeSize*3}, {-2*\verticalSpacing}) {5};
\draw[treeEdge] (n-1-1) -- (n-2-3);
\draw[treeEdge] (n-0-0) -- (n-1-1);
\end{tikzpicture}\end{center}
\endgroup\par

\end{document}
